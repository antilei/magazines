\documentclass[UTF8,12pt,oneside]{ctexbook}
\usepackage{graphicx}
\usepackage{xeCJK}
\usepackage{indentfirst}
\usepackage{titletoc}
\usepackage{fancyhdr}
\usepackage{fontspec}
\usepackage{color}
\setmainfont{Times New Roman}
\pagestyle{plain} % 此处为fancy时有页眉
\title{\textbf{\fontsize{42}{84}{反雷杂志 \\[15] Anti-Lei Magazine}}}
\author{\Large \kaishu 雷氏力学吧、太差太差吧内部资料 \\[8] \Large \kaishu 《反雷杂志》编辑部}
\date{\huge 第5期 (2021年8月)}

\begin{document}

    \large
    
    % 目录配置,请勿改动
    \setcounter{secnumdepth}{-2} 
    \setcounter{tocdepth}{1}
    \titlecontents{section}[16pt]{\addvspace{2pt}\filright}
    {\contentspush{\thecontentslabel\hspace{0.8em}}}
    {}{\titlerule*[8pt]{.}\contentspage}

    \tableofcontents

    \chapter{藏字概论}
    \hfill{\Large ——谈谈《咏雷》中的“内鬼”(一)}
    ~\\
    \begin{center}
        Mono6
    \end{center}

    \newcommand{\red}{\textcolor{red}}

    \subsection{摘要}
        本文介绍了雷绍武《咏雷集》中藏字作品的历史,系统地总结了藏字的常见类型,如直线藏字、对角线藏字、马步藏字、特殊藏字等,创新性地提出了“矢量描述法”来描述一些典型的藏字,基于抽样统计数据和实例分析了藏字作品的命运,并给出了藏字作者的反制手段。
    
    \subsection{关键词}
        《咏雷集》\ \ \ \ 藏字\ \ \ \ 反雷\ \ \ \ 雷绍武\ \ \ \ 街角

    \section{一、藏字作品简史}

    \subsection{最早的藏字作品}
        最早的藏字作品可以追溯到《咏雷·二八》,这其实是一首反雷作品,作者的贴吧用户名为“云淡风清”:

        \begin{center}
            \heiti [七律] 咏雷·二八

            \songti 作者\ \ 云淡风清
            
            \kaishu
            \red{雷}\footnote{本文中一律使用彩色标注藏字内容。}崽吹牛嘴碎烦,

            \red{神}言运动力空前。
            
            \red{大}篇鼓捣三心术,
            
            \red{傻}话纷飞四海间。
            
            \red{民}意久塞谁叹气,
            
            \red{科}盲跋扈醒来难。
            
            \red{必}将臆想传天下,
            
            \red{亡}灭魂灵不可还。

        \end{center}

        这是一首简单的藏头作品,藏字的内容也是直接辱骂:雷神大傻民科必亡。值得一提的是,从现存的《咏雷集》来看,这是雷绍武收录的第一首其他人的作品。因此,雷绍武很可能是第一次遇到这种情况。他非常愤怒,不仅小改了这首诗,把藏头变成了“雷神大智民科必胜”(《咏雷·二九》),并且在之后的一系列诗歌中多次辱骂这首诗的作者云淡风清。(篇幅有限,此处不赘述)

        之后,雷绍武和一些咏雷派似乎受到了启发,写了较多的含有咏雷藏字(如“雷神伟大”,《咏雷·八三,八四》)的咏雷诗;当然反雷派也纷纷效仿。反雷派作出藏头诗后,一部分咏雷派会当面拆穿,这使得这个阶段的反雷藏字诗实际上相当少。但是,雷绍武自己识别藏字的能力是很差的,例如:

        \begin{center}
            
            \heiti 咏雷·二九三(原诗)\footnote{这里说是“原诗”,是因为现在收录的版本中被雷绍武发现了藏字而小改了,破坏了原有的藏字内容。后同。}

            \songti 作者\ \ 万用壶
            
            \kaishu
            \red{雷}公绍武智无边,

            \red{神}言力学最前沿。
            
            \red{弱}傻不识运动力,
            
            \red{智}者笑看风云间。
            
            \red{民}心久塞公多叹,
            
            \red{科}奴跋扈训道难。
            
            \red{必}将运力传天下,
            
            \red{亡}吾身兮亦不还。

        \end{center}

        这分明就是小改了前面的《咏雷·二八》,但是,雷绍武看不出这个他曾经踩过一次的坑。他在这首诗下评论道:“谢谢。好诗啊,很有水平。”后来在吧友的反复提醒下,雷绍武才改了这首诗。

        值得注意的是,有一部分人认为最早的藏字作品是雷绍武自己写的《咏雷·一》:

        \newpage

        \begin{center}
            
            \heiti 咏雷·一(原诗)

            \songti 作者\ \ 雷绍武

            \kaishu
            平地\red{一}声雷,

            惊醒\red{梦}中人。

            梦幻\red{全}扫光,

            人人\red{能}思维。

        \end{center}
        即藏字“一梦全能”。笔者认为这是不恰当的,因为这是雷绍武自己写的作品,他不可能自己藏入一个反雷程度这么强的语句。当然,雷绍武自己也是不承认的。因此笔者认为,“一梦全能”仅仅是巧合,而不是有意的藏字。

    \subsection{藏字作品的发展}
        随着时间的推移,雷绍武也能够发现大部分藏头了,这迫使吧友在创作反雷藏字诗时必须考虑更加复杂的藏字。据考证,最早出现的复杂藏字应当是对角线藏字和反对角线藏字,例如银海君的以下作品:(这两首诗作于2019年9月末、10月初)

        \begin{center}
            
            \heiti 咏雷·四二七(原诗)
            
            \songti 作者\ \ 银海柔光透冰河
            
            \kaishu
            重云裂落一声\red{雷},
            
            破除邪说歪\red{理}摧。
            
            成电教授\red{论}战起,
            
            混淆对\red{错}与是非。
            
            牛顿\red{漏}洞绍武补,
            
            四\red{百}诗扬民科威。
            
            \red{出}征破敌在今日,
            
            名满中华舍我谁。

        \end{center}
    
        \begin{center}
            
            \heiti 咏雷·四二八(原诗)
            
            \songti 作者\ \ 银海柔光透冰河
            
            \kaishu
            \red{雷}雨天落注岷江,\red{一}生求实道路长。
            
            爱\red{氏}谬论光速定,妖\red{派}妄想丑态彰。
            
            牛顿\red{力}有三定律,推导\red{胡}扯错主张。
            
            我劝大\red{学}信绍武,创新真\red{言}进课堂。

        \end{center}
        
        说到银海君的藏字作品,就不得不提另一个前知名吧友:猴山7。猴山7自称挺雷派,他非常热衷于去找吧友咏雷诗中的藏字。事实上,他确实找出了很多藏字(例如上面的这两首就是他找出的),但是更多的是牵强附会地去挑字,还给他自认为的“藏字者”扣上反雷派的帽子,辱骂其为“两种人”。许多主要的咏雷创作者都被他不停地攻击和纠缠过(而银海君就是最大的受害者之一),严重破坏了吧内的气氛和秩序。2020年1月,猴山7被以银海君为首的几人公审驱逐雷氏力学吧和反民科吧。

        之后,在2020年末、2021年初,街角等人开创了更为复杂的藏字方法,如马步藏字法等。这些藏字雷绍武很难发现,因此众人纷纷模仿。事实上,现存《咏雷集》中,大部分未修改的藏字作品都是马步藏字等更复杂的藏字。可以说,街角等人的创新,掀起了咏雷藏字的高潮。例如,以下这首作品,藏字为“我是反雷派,雷猴快进盒”:

        \begin{center}
            \heiti 咏雷·九五四\ \ 临江仙慢

            \songti 真理守护者~街角的城事

            作者:国外雷氏力学研究人员沃兹基先生

            2021.3.19 9:25:52
            
            \kaishu
            无私忘\red{我}为真理,
            
            求\red{是}问非心如玉。
            
            迷途难\red{反}科奴愚。

            天\red{雷}扫谬论,\phantom{。。}

            自成一\red{派}传奇。\phantom{。}
            
            十年惊\red{雷}震天地,
            
            王\red{侯}将相\footnote{王侯将相:指居于统治地位的官科。}临大敌。
            
            大道明\red{快}民智启。
            
            前\red{进}复创造,\phantom{。。}
            
            恰如长\red{河}不息。\phantom{。}

        \end{center}
            
        目前,藏字技术继续发展,还出现了阶梯形、V形、倒藏等新型藏字法。作为咏雷诗歌的重要组成部分,藏字值得我们继续研究。

    \section{二、藏字的主要类型}
        \subsection{直线藏字}

        指藏字内容直接排成一竖列,主要是藏头(即藏字内容为每行的第一个字),如前面提到的“雷神弱智民科必亡”。又如:

        \begin{center}
            \heiti 咏雷·九九九

            \songti 作者\ \ 氢氦锂铍硼

            2021/4/2 23:59:33

            \kaishu
            \red{平}淡千载科学史,

            \red{地}覆天翻亦几回。
            
            \red{一}日智者降天府,
            
            \red{声}讨管科真理归。
            
            \red{雷}鸣电闪寰球震,
            
            \red{惊}天撼地定万规。
            
            \red{醒}悟万民皆拥护,
            
            \red{梦}中官科何倾颓。
            
            \red{中}华大地尽拥趸,
            
            \red{人}民汪洋凯歌吹。
            
            \red{梦}魇终除人智开,
            
            \red{幻}象再难侵思维。
            
            \red{全}球各地皆响应,
            
            \red{扫}清学阀承天威。
            
            \red{光}茫普照天下土,
            
            \red{人}杰地灵乐山惠。
            
            \red{人}尽皆知雷绍武,
            
            \red{能}文善武万古垂。
            
            \red{思}想不绝如岷江,
            
            \red{维}民所止皆拥雷!

        \end{center}

        有时候还会出现双列直线藏字,如:(这首诗未收录,作者为Mono6)

        \begin{center}
            \kaishu
            \red{必}将咏\red{雷}慢慢品,
            
            \red{须}把运\red{力}久久研。
            
            \red{严}推细\red{理}出高论,
            
            \red{惩}办谬\red{论}昭普天。

        \end{center}

        直线藏字很容易被雷绍武发现,因此《咏雷集》中直线藏字以咏雷的为主,反雷的极少。

        \subsection{斜线藏字(象步藏字)}

        指沿斜线(45度角)排列藏字内容的藏字方法,类似于国际象棋中“象”(Bishop)的走法,故又称“象步藏字”。主要包括长斜线藏字和斜折线藏字。

        (1) 长斜线藏字:指沿同一条大斜线方向不变的藏字,包括沿对角线和反对角线的藏字,如前面提及的两首银海君的作品。又如这一首“穿越边界”的长斜线藏字:

        \begin{center}
            \heiti 咏雷·九四三
            
            \songti 街角的城事
            
            作者:国外雷氏力学研究人员沃兹基先生
            
            2021-03-13 13:46

            \kaishu
            真理在手何所惧?
            
            心底\red{无}私天地宽。
            
            不\red{耻}下问十年苦,
            
            \red{至}圣先师万代传。
            
            大音无声出太\red{极},
            
            大道无形成\red{雷}理。
            
            中华民科\red{绍}天道,
            
            经文纬\red{武}造传奇!

        \end{center}
        
        (2) 斜折线藏字:指沿斜线的方向,一步一折的藏字,如:

        \begin{center}
            \heiti 咏雷·五五一\ \ 雷霆万钧
            
            \songti 作者\ \ 大地重归寂灭

            \kaishu
            \red{雷}声一响惊天地,

            牛\red{氏}谬论无处藏。

            \red{力}挽狂澜传真理,

            求\red{学}十载岁月长。

            \red{笑}看人间风波起,

            谁\red{料}反雷太猖狂。

            \red{百}炼成钢何所惧。

            走\red{出}乐山美名扬。

        \end{center}

        \subsection{马步藏字和广义马步藏字}

        马步藏字指沿类似于国际象棋中“马”(Knight)的走法排列藏字内容的藏字方法,,即在下一行的对应字的左侧或右侧第二个字藏下一个字。也包括沿同一方向藏字和折回藏字两种。这种藏字难以被雷绍武发现(当然有人拆穿就另当别论了),因此数量最多,在前面已经用街角的诗歌举例说明了。又如:

        \begin{center}
            \heiti 咏雷·九0七(原诗)
            
            \songti 街角的城事
            
            2021/2/17 0:13:44
            
            国外雷氏力学专家沃兹基先生复作咏雷一首,请我代为发布。以下为原作:

            \kaishu
            乐山平地起惊\red{雷},

            运动力学\red{绍}天规。
            
            文治\red{武}功谬论去,
            
            \red{是}古非今真理回。
            
            手握真理万\red{民}\footnote{穿梭藏字,上一句“是”字左数两格,就穿越到从最右侧开始的第二个字}至,
            
            脚踏官\red{科}四力归。
            
            苍\red{狗}白云道至简,
            
            雷力等于K M V!

        \end{center}

        \newpage

        \begin{center}
            \heiti 咏雷·一三0五\footnote{藏字为谐音的“雷绍武显然进盒中”。为了防止雷绍武发现,发布时采用了两句一行的排列。}
            
            \songti Mono6
            
            2021/8/4 11:13

            \kaishu
            \red{雷}门出智者,
            
            博学\red{少}有为。
            
            \red{五}谷化真理,
            
            四力\red{显}神威。
            
            \red{燃}毁官科卷,
            
            铸成\red{进}步碑。

            \red{何}方寻伯乐?
            
            电子\red{终}相随!

        \end{center}

        而“广义马步藏字”,指的是比马步藏字横向跨度更大的藏字,比如下一个字在这个字的下一行的左(右)数第3,4,5,……个字(方便起见,我们把这个数称为跨度)等。有时候还会将几种跨度混用,这也可以属于广义马步藏字。如:

        \begin{center}
            \heiti 咏雷·一0四九

            \songti 真理出乐山

            2021/4/22 10:00:29

            作者:国外雷氏力学研究人员沃兹基先生

            \kaishu
            忆昔\red{雷}吧全盛日,

            \red{神}贴未曾有尽时。
            
            纵有\red{无}知官科党,
            
            \red{耻}见无德反雷狂。
            
            蛆虫虽\red{快}终无理,
            
            \red{点}金乏术世人弃。
            
            以退为\red{进}民智启,
            
            \red{合}胆同心造传奇!

        \end{center}

        这是混用跨度2和跨度3的广义马步藏字。

        



\end{document}