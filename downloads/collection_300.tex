
%%%%%下面这些都是全文的配置,别乱动就是了
\documentclass[UTF8,12pt,oneside]{ctexbook}
\usepackage{graphicx}
\usepackage{xeCJK}
\usepackage{indentfirst}
\usepackage{titletoc}
\usepackage{fancyhdr}
\usepackage{fontspec}
\usepackage{geometry}
%\geometry{left=2.5cm,right=2.5cm,top=2.5cm,bottom=3.0cm}
\setmainfont{Times New Roman}
\pagestyle{plain} % 此处为fancy时有页眉
%%%%%上面这些都是全文的配置,别乱动就是了

%%%%%这是封面页
\large
\title{\fontsize{42}{84}\textbf{反雷集1\textasciitilde 300}}
\author{\LARGE 反雷社}
\date{\LARGE 2021年7月27日}

%%%%%这里定义了宏
\def\pau#1{\begin{center} {#1} \end{center}} %第一个宏:诗的作者
\def\pst#1{\begin{center}\Large \kaishu {#1} \songti \large \end{center}}%第二个宏:诗的副标题
\def\poem#1#2{\section{#1}\pau{#2}} %第三个宏:标题+作者
\def\poemwithst#1#2#3{\section{#1}\pst{#2}\pau{#3}} %第四个宏:标题+副标题+作者
\def\shanglian#1#2
    {\noindent{\textbf{上联:}({#1})}

    \begin{center}
        {#2}
    \end{center}
    }

\def\xialian#1#2
    {\noindent{\textbf{下联:}({#1})}

    \begin{center}
        {#2}
    \end{center}
    }

\def\li{\setlength\parindent{4em}}
\def\lid{\setlength\parindent{5em}}
\def\lidl{\setlength\parindent{6em}}
\def\lidli{\setlength\parindent{7em}}
\def\lidlid{\setlength\parindent{8em}}
\def\lidend{\setlength\parindent{2em}}

\begin{document}
    \large
    
    \maketitle %生成封面页
    
    %%%%% 目录配置,请勿改动
    \setcounter{secnumdepth}{-2} 
    \setcounter{tocdepth}{1}
    \titlecontents{section}[16pt]{\addvspace{2pt}\filright}
    {\contentspush{\thecontentslabel\hspace{0.8em}}}
    {}{\titlerule*[8pt]{.}\contentspage}

    \chapter{前言}

    热烈祝贺反雷诗歌突破300首!

    感谢朋友们的大力支持和积极参与,反雷诗歌已经突破300首了。
    
    这是完全出乎我们的意料的。我们原本打算零零散散写几首就差不多了。后来50首了,再后来100首了,再后来200首了,现在是超过300首了。我们真的非常感谢大家!

    我们都不是专业的诗人,也不需要用专业的诗歌水平去要求她。自然之美,内心的美,比那种矫揉造作的美更美。

    反雷诗歌不是对雷绍武的个人攻击,也不是激化与咏雷派的矛盾。而是对雷的学识,雷的精神,雷的谬论,雷的态度,雷的影响的反对。是对雷氏理论,雷氏精神的反对。

    因时间和水平有限,本诗歌集在文学水平、排版编辑等各方面都有诸多不足,敬请批评指正。
    
    \tableofcontents

    \chapter{}
    创作《反雷》诗歌的目的是和雷绍武的《咏雷》系列作中门对狙。
    体裁和内容不限,只需做到以下几点即可:
    
    1. 诗歌不能含有对雷绍武本人及其家人的辱骂,当然如果你藏字辱骂那就无所谓了。
    
    2. 不允许对其他吧友进行辱骂。
    
    3. 不要夹带私货,比如抽象,深深这些。
    
    \begin{flushright}
        \kaishu{——RobL61,2021年4月25日于太差太差吧}
        \songti
    \end{flushright}
    
    \section{1 无题}
    \begin{center}
        RobL61
        
        ~\\
        白丁愚民如风散,鸿儒正论难可替。
        
        牛顿果树悟引力,伽氏斜塔知落体。
        
        伦蒂尼恩喜雨济\footnote{伦蒂尼恩:即伦敦,与上联牛顿相呼应。},阿诺河畔波涛汐\footnote{阿诺河:比萨城的母亲河,伽利略是比萨人。}。
        
        乐山老朽欲撼树,孰知真理安可戏?

    \end{center}
    
    \newpage
    
    \section{2 无题}
    \begin{center}
        带带帅师兄258
        
        ~\\
        雷门不幸出佞子,绍述邪义秉歪理。
        
        武薄文浅高山鼓,是异非同石妇逼。
        
        个中诸多荒唐言,老眼残躯犹自喜。
        
        白日蜀犬夜吴牛,痴道虚妄运动力。
        
        ~\\
        
    \end{center}
    
    \section{3 打油诗}
    \begin{center}
        RobL61
        
        ~\\
        可怜绍武命已残,天天被人当猴看,
        
        早年不入八二六,今朝何会如此惨?
        
        ~\\

    \end{center}
    
    \section{4 无题}
    \begin{center}
        RobL61
        
        ~\\
        哈佛牛津占前沿,北大清华涨国颜。
        
        乐山绍武惶度日,怎见恩师杨正贤?
        
        ~\\

    \end{center}
    
    \newpage
    
    \section{5 无题}
        \begin{center}
            带带帅师兄258
            
        \end{center}
        
        雷绍武!
        
        我原以为你身为民科老人,来到贴中,面对诸多网友,必有高论,没想到竟说出如此科妄之语!
        
        我有一言,请诸位静听。
        
        如今科学昌明之时,民智大开,百业已兴,日新月异,突飞猛进。盛世之中,生物,数理,化学等接踵而起,神六升空,蛟龙下海。
        
        因之,学府之上,人才辈出;科院之间,发明不断。以至雄心壮志之辈巍巍当朝,经天纬地之才纷纷秉政。以致社稷变为富饶,苍生安享太平之乐!值此国幸之际,雷绍武又有何作为?
        
        你世居巴蜀之隅,初举乐职入仕;理当传道授业,桃李天下;何期虚词诡说,背道而驰!罪恶深重,天地不容!
        
        无耻老贼!岂不知天下之人,皆愿生啖你肉!安敢在此饶舌!
        
        今幸天意欲绝妖邪,我等网友于贴吧扛起大旗。我今奉天之命兴师讨贼。你既为宵小之辈,只可潜身缩首,苟图衣食,还敢在我军面前妄称天数?
        
        皓首匹夫!苍髯老贼!你即将命归于九泉之下,届时,有何面目见你家二十四代祖宗?!
        
        二货贼子!你枉活七十有二,一生未立寸功,只会摇唇舞舌,欺世盗名!一条断脊之犬,还敢在我军阵前狺狺狂吠?
        
        我,从未见过如此厚颜无耻之人!!!
        
    \newpage
    
    \section{6 无题}
    \begin{center}
        带带帅师兄258
        
        ~\\
        十年间,铁打的营盘,流水的兵。
        
        每一年,送走一批又一批,
        
        本科生,研究生。
        
        雷老师,运动力,磁相连,雷电子,
        
        搞笑理论出不穷。
        
        咏雷诗,唱一遍,吟一首,
        
        藏字藏头说不停。
        
        众吧友,笑哈哈,哈哈笑,
        
        阴阳怪气问不明。
        
        光阴过,青丝白,筋骨轻,
        
        科学一梦尤未醒。
        
        俱往矣,忆扣饭,校长请,
        
        正贤音容满别情。
        
        再回首,古稀年,天地清,
        
        民科绍武最“英雄”!

    \end{center}

    \newpage
    
    \section{7 无题}
    \begin{center}
        RobL61
        
        ~\\
        雷公居于岷江畔,不识洋文不善汉。
        
        前有成电肚子疼,后有街角藏字繁。
        
        春秋大梦仍未醒,自我高潮独一份。
        
        雷吧已成动物园,使人欢喜使人叹。

        ~\\

    \end{center}

    \section{8 天净沙\ \ 反雷}
    \begin{center}
        RobL61
        
        ~\\
        
        大佛岷江嘉州,
        
        川渝乐土无忧,
        
        川大井盖藏猴\footnote{井盖藏猴:指“文革”时期雷绍武武斗失败躲进井盖的事情。}。
       
        明珠暗投\footnote{明珠暗投:指雷绍武作为川大高材生,本有一番作为,却去参加红卫兵组织。},
        
        雷绍武在缩头\footnote{雷绍武在缩头:同“井盖藏猴”。}。
    \end{center}
    
    \newpage

    \section{9 江城子·评雷}
    \begin{center}
        Mono6
        
    \end{center}
       
       乐山绍武常发狂,摸零线,拭眼眶。夜起如厕,静人而动床\footnote{静人而动床:即“人静而床动”。}。约d开创运动力,嘲管科,必将亡。    
       
       但惜年少学业荒,逾古稀,似空囊\footnote{逾古稀,似空囊:年龄已经七十有余,思想和学识却依然空如皮囊。}。不学无术,所思皆为妄。劝君勤学以为戒,莫见笑,于四方。
    ~\\

    \section{10 采桑子}
    \begin{center}
        RobL61
    \end{center}
       
        乐山雷公甚好梦,普天皆雷。牛顿算谁?咏雷反雷百事非。
       
       反对雷理皆无知,“四无”人哀\footnote{“四无”人哀:真心想让雷绍武不再沉湎于歪理,却被其认为是四无人的反雷电子。}。反串狂嗨\footnote{反串狂嗨:自以为为了雷绍武好,却仅仅让他出更多笑话。当雷绍武又提出歪理时,反雷派会纠正他,而所谓拥雷派却在旁边把他当笑话。},只愿绍武快醒来。
       
    \newpage 
    
    \section{11 愚翁歌}
    \begin{center}
        RobL61
        
    \end{center}
    
        四川愚翁,世居川渝盆地中。吾辈虚度若干年,愚翁身世何其坎。四廿年前降于世,早年身世困难重。跃进时期食草树\footnote{跃进时期食草树:雷绍武曾言:“一个人的时候……对于一个嚼过草根树皮的人来说,已经是很幸福的事了。”(《我的生活很简单》)因此我们有理由认为雷绍武在大跃进时期遭受过饥饿。},反右浪潮恩师护。英勇男儿,怎为鹰犬和走狗?
        
        翻天覆地大革命,川大东红八二六。不学无术去武斗,一三二厂井盖愁。革命浪潮既已然,平平淡淡数十年。
        
        新生网络,唤醒老翁再激情。早年肮脏也就罢,运动谬论晚节疤。唇枪舌战论吧友,慷慨激昂斥管科。雷氏力学新天地,但被别人看猴戏。
        
        愚翁绍武,何其壮阔一辈子。无知无德两种人,说的就是你自己。八二六,跳井盖。辩吧友,好气概!论谬理,多可哀!少受骗,多思考。勤学习,莫卖老。多沉淀,少暴躁。愚翁蜕变,指日可见。
        ~\\
        
    \newpage

    \section{12 破阵子·为雷绍武附悲词以寄}
    \begin{center}
        RobL61
        
    \end{center}
       
        宅里挑镜看贴,梦回川大时代。慷慨激昂闹革命,狼狈逃窜跳井盖,差点命不在。
        
        而今网络时代,对线对不过来。不懂就说看不清,无能狂怒净被踩,只能让人哀。
        ~\\

    \section{13 天雷}
    \begin{center}
        \Large
        \kaishu{小改郭沫若《天狗》}
        \songti
        \large
        
        带带帅师兄258
        
        ~\\
        我是雷绍武呀!
        
        我把d来约了,我把零线摸了,
        
        我把一切的星球来吃了,
        
        我把全宇宙来吞了。
        
        我便是我了!
        
        ~\\
        
        我是月的光,我是日的光,
        
        我是一切星球的光,我是X光线的光,
        
        我是全宇宙的运动力的总量!
        
        ~\\
        
        我飞奔,我狂吠,我燃烧。
        
        我如电子火花一样地燃烧!
        
        我如暮年老犬一样地狂吠!
        
        我如峨眉猿猴一样地飞跑!
        
        我飞跑,我飞跑,
        
        我飞跑,我剥我的皮,
        
        我食我的肉,我嚼我的血,
        
        我啮我的心肝,
        
        我在我神经上飞跑,
        
        我在我脊髓上飞跑,
        
        我在我脑筋上飞跑。
        
        ~\\
        
        我便是我呀!
        
        我的我要爆了!

    \end{center}

    \newpage
    
    \section{14 蝶恋花·览雷力吧数月有感}
    \begin{center}
        RobL61
        
    \end{center}
       
        乐山老翁创雷力,打开贴吧,大笑生天际。但嘲管科不量力,何与雷绍武为敌。
        
        雷力何与管科比?无德无知,荒唐运动力!若此谬论即雷理,那我真心看不起\footnote{那我真心看不起:改编自雷绍武名言:这就是你们“管科”的水平吗?我看不起啊!}。
    
    ~\\
    \section{15 水调歌头·谬论何时绝}
    \begin{center}
        RobL61
        
    \end{center}
    
        谬论何时绝?追忆中世纪。地心荒唐邪说,毒害科学界。再看雷氏宇宙,公转五百万年\footnote{公转五百万年:雷绍武认可过的根据雷氏理论运算的公转周期。},今朝是何月?雨滴会致命\footnote{雨滴会致命:雷绍武曾认为:高处以高速下落的雨滴有打死人的风险。},雷公颅开裂。
        
        大笑毕,颇感慨,令人嗟。心有不甘,神州何时出人杰?贴吧民科绍武,庙堂尸位素餐,都已至耄耋。劝君知进退,免得晚节劣。
        ~\\
    
    \newpage
    
    \section{16 古体诗}
    \begin{center}
        带带帅师兄258
        
        ~\\
        山川同作证,贼徒肆狂猖。歹徒击破鼓,沐猴梦己强。
        
        含沙欲射影,丧心连病狂。诡计言隐晦,弄技舞刀枪。
        
        狂犬吠红日,无损日光芒。蚍蜉撼大树,怒斥不自量。
        
        江河逆流滚,岂阻巨舟航?枉活七十七,兔尾岂可长!
        
    \end{center}
    
    \section{17 论雷氏理论的种种错误}
    \begin{center}
        RobL61
        
        ~\\
        先有雷神后有天,两球落地大球先。
        
        雨滴下落会致命,地球公转百万年。
        
        行星绕柱作自转,乳雷皆为不刊论。
        
        雷学没有参考系,人都只在袜子里。
        
        重力加速四点九,实验视频皆为戏。
        
        雷的理论皆为真,反对他的就是锑。
        
        发光发热皆太阳,我们都被烧成炭。
        
        川大中文系毕业,危言危行太难堪!
        
        麦瑟尔夫沃兹基,不会洋文太为难。
        
    \end{center}
    
    \section{18 五绝·雷绍武解封又被封有感}
    \begin{center}
        RobL61
        
        ~\\
        绍武刚出笼,又欲嬉于众。
        
        香蕉虽珍馐,应以晚节重!
        
        
    \end{center}
    
    \section{19 进盒雷}
    \begin{center}
        \Large
        \kaishu{改编自《魏书·孝静纪第十二》}
        \songti
        \large
        
        RobL61
        
    \end{center}
    
        吧主尝对雷曰:“臣劝雷神退。”雷不悦,曰:“自古无不亡之吧,我亦何要退!”吧主怒曰:“雷!雷!进盒雷!”吧主遂封雷十天,奋衣而出。
        
        
    \section{20 七绝·反雷}
    \begin{center}
        RobL61
        
        ~\\
        巍巍峨眉俯瞰川,涛涛岷江滚河岸。
        
        鼎堂诗词何其美\footnote{鼎堂:即郭沫若,四川乐山人},绍武民科使人叹。
        
    \end{center}

    \newpage

    \section{21 雷理肯定灭亡/真理不会灭亡}
    \begin{center}
        \Large \kaishu
        改编自波兰国歌《波兰不会灭亡》
        \songti \large
        
        RobL61
    \end{center}
       
    \begin{center}
        雷氏谬论肯定灭亡,只要我们尚存。
    
        举起真理,打倒雷力,直到科学泽润。
        
        前进,前进,艾萨克·牛顿!
        
        从不列颠到四川
        
        在真理领导下,我们将亲如一家。
        
        ~\\
        我们将跨过经典力学,走向微积分。
        
        成为真正科学人。
        
        布鲁诺已经告诉我们,
        
        怎样去回击谬论!
        
        前进,前进,艾萨克·牛顿!
        
        从不列颠到四川!
        
        在真理领导下,我们将亲如一家!
        
        ~\\
        正如伽利略到比萨斜塔,
        
        造成伪科学的坍塌!
        
        为了保卫我们的新芽,
        
        我们将雷理打。
        
        前进,前进,艾萨克·牛顿!
        
        从不列颠到四川!
        
        在真理领导下,我们将亲如一家!
        
        ~\\
        教授对学生眼含喜情,
        
        激动的说:
        
        听啊,我们新一代的栋梁。
        
        正把真理越辩越明。
        
        前进,前进,艾萨克·牛顿!
        
        从不列颠到四川!
        
        在真理领导下,我们将亲如一家!
        
    \end{center}
    
    \section{22 猴吟}
    \begin{center}    
        雷绍夫
    \end{center}
        
        岷江二月三月天,千峰万壑雷光开。有如地球公转百万载,又似雷氏客机喷水来。半夜惊厥床动人不动,半天唯见黑色火焰烧熊熊。管科学阀震簌簌,缘是雷猴下凡来。
        
        面红耳赤大马猴,无耻无赖八二六。幸得慧根能人言,又有颅内新理论。自诩慧根冠今世,又言理论胜牛顿。贴吧开宗明雷义,荒谬不堪败人意。
        
        网友惜其高寿者,以为有知可教说。然猴不自怜而进盒,不听劝而辱骂管科。整日孤影自怜惜,作咏雷而自麻痹。终成可笑可悲事,恶名传遍雷力群。
        
        呜呼哀哉,雷猴本无知,然其通人言,能思辨,本可教化。而其夜郎自大,无耻无赖,不听劝谏,终成笑柄,岂不悲哉。
        
    \section{23 雷吧铭}
        \begin{center}
        \Large \kaishu
        改编自刘禹锡《陋室铭》
        \songti \large
        
        RobL61
        \end{center}
        
        吧不在大,有梗则名。园不在广,有猴则灵。斯是雷吧,惟绍武明。荒唐运动力,雨滴会致命。谈笑有“真理”,往来无“白丁”。可以写咏雷,阅雷经。无烂梗之乱眼,无上班之劳形。野生动物园,乐趣十分顶。绍武云:本吧是一个舞台,各式各样的人都可以表现!
        
        ~\\
        
    \section{24 水仙子·涌雷和反雷}
        \begin{center}
        RobL61
        \end{center}
        
        一只绍武谬理谈,无数反雷推其翻,雷绍武埋怨智商淡。看雷猴叫再三,毁惊雷智商八千\footnote{毁惊雷智商八千:改编自张养浩咏江南名句:卷香风十里珠帘。原顺序为“惊雷毁八千智商”,即雷绍武的理论毁了八千余雷吧吧友的智商,让他们把自己和雷绍武置于一个水平。}。反雷派斥谬言,涌雷派在狂欢。一目了然!

    \newpage
        
    \section{25 半打油诗·览雷绍武在QQ群发言有感}
        \begin{center}
        RobL61\footnote{谨以此诗纪念喷水式飞机的没落。}
        \end{center}
        
        \begin{center}
        欲问绍武喷水机,一顿废话令人嬉。
        
        飞机喷油速度快,神器绍武来普及。
        
        哪知翻脸不认账,反正和雷没关系。
        
        年少疯狂既已去,晚节毁于运动力!
        ~\\
        \end{center}
        
    \section{26 短文·雷绍武的辩论方式}
        \begin{center}
            RobL61
        \end{center}
        
        别人长篇大论的质疑雷绍武的歪理,雷绍武就抓住其中一个点来以提问的方式解决问题,这种人我们俗称杠精。
        当别人引用一些雷绍武从来没有见过,了解过的概念时,雷绍武不会放下身段,虚心求教。
        
        反而问:“有人证明xx的存在吗?有视频证据吗?证明xx是人云亦云,胡说八道的。”
        
        当别人逻辑严密或者用雷绍武完全一无所知的概念和雷绍武辩论,那么雷绍武就会装死,或者来一句万能的“不和无知无赖纠缠”来结束话题。
        
        证明:雷绍武826作威作福惯了,根本经不起严谨的辩论,别人没占上风时就大肆抬杠。别人若是占了上风,那就像攻打132时一样,把他可怜的小脑袋缩进井盖里。只能证明826本性难移。
        
    \section{27 念奴娇·反雷}
        \begin{center}
            赵明毅
        \end{center}

        寻常一日,在花果山里,有猴惊起。自恃清高狂辱骂,创立邪说歪理。无耻无德,行将就木,狗眼瞧人低。饱尝耻笑,自将为人所弃。 
        
        老狗三万学说,细读只见,放尽乾坤屁。奇数碳环为链状,球大更先接地。零线无流,床移人静,分子粘接力。十年一日,永存雷老猴戏。
        
    \section{28 无题}
        \begin{center}
            析境
        \end{center}
            
        \begin{center}
            雷氏力学的歪理开始宣扬,
            
            如今雷绍武竟如此猖狂!
        
            ~\\
            咏雷的诗歌已经有了一千,
            
            不知多少都是藏字之言。
            
            ~\\
            藏他半夜起床床动人止,
            
            藏他自己只有唯一位置。
            
            ~\\
            藏他高喊着$L=kmv$,
            
            藏他那完全错误的运动力。
            
            ~\\
            愿雷猴不要就此停息,
            
            我们还等着看更多猴戏!
        \end{center}
        
        \section{29 真理赋}
        \begin{center}
            肚子疼教授/八二六
        \end{center}
            
        \begin{center}
            雷所追求为真理,真理也即一泡屎。
        
            浑身上下即真理,雷猴此论令人嬉。
        
            着意辱骂雷吧友,泼妇骂街空至极。
            
            苦口婆心杜自誊,无知雷猴岂能意?
        \end{center}
        
        \section{30 扫雷·八二六}
        \begin{center}
            肚子疼教授/八二六
        \end{center}
            
        \begin{center}
            平地一牛顿,雷猴言不择。
            
            为何雷不动?原是太无德。
        
            盒动雷不动,八二六进盒!
        \end{center}
        
        \section{31 无题}
        \begin{center}
            yogцгт
        \end{center}
            
        \begin{center}
            无知无德创雷力,众人跑来看猴戏。
            
            唯有雷老颅中研,错把误论当真理。
        \end{center}
        
        \section{32 咏雷·八二六}
        \begin{center}
            中科院理论二所杜自誊教授/八二六
            
            一三二〇年八月二十六日
        \end{center}
            
        \begin{center}
            雷氏力学,一派胡言,
            
            以建粪土,以建进盒。
            
            被骗多次,Dr.Oligayeating,
            
            夙夜匪懈,礼据四无。
            
            矢谨矢勇,必德必忠,
            
            一心一德,贯彻始终,
            
            雷氏力学终将灭亡。
            
            雷老狗终将进盒。
        \end{center}
        
        \section{33 扫雷波}
        \begin{center}
            肚子疼教授/八二六
        \end{center}
        
        \begin{center}
            乐山井盖雷出盒,
            
            吧友纷纷来看戏,
            
            世间尺度雷意淫,
            
            自然真理难会意。
        \end{center}
        
        \newpage
        
        \section{34 无题}
        \begin{center}
            人不动
        \end{center}
        
        \begin{center}
            老而不死而为贼,绍武多年总无为。
            
            敢将谬论公堂上,不知运动怎真伪。
            
            四无绍武无真意,口吐真理留芳菲。
        \end{center}
        
        \section{35 反雷·小改咏雷〇〇一}
        \begin{center}
            RobL61
        \end{center}
        
        \begin{center}
            平地一声雷,
            
            笑醒梦中人。
            
            智商全扫光,
            
            没人能思维。
        \end{center}
        
        \section{36 无题}
        \begin{center}
            渺天汉
        \end{center}
        
        \begin{center}
            反雷团,雷公憨。只因绍武太脑瘫。
            
            劝涌雷,应知惭。雷猴无知出逆言。
            
            雷空怒,雷徒恼。无能狂怒不要脸。
            
            挺雷派,尽反雷。雷氏理论在猴山!
        \end{center}
        
        \newpage
        
        \section{37 定风波}
        \begin{center}
            yogцгт
        \end{center}
        
        狗屁不通创雷力,错把误论当真理。古稀之年耍猴戏。无知,何来自信反正理。
        
        颅中实验验道理,蒙昧,若无实践怎证理。邪说歪理无验证,无知,应触零线治愚病。
        ~\\
        
        \section{38 浣溪沙}
        \begin{center}
            yogцгт
        \end{center}
        
        床动人静是为何。两球同落大先着。狗屁不通的玩意。
        
        零线里面无电流,地球公转十万年。世间永存雷猴戏。
        ~\\
        
        \section{39 菩萨蛮}
        \begin{center}
            yogцгт
        \end{center}
        
        无知却要创雷力。理论有误劝不听。参考系有误。床动人不动。
        
        手搓电火花。分子有两极。如泼猴无礼。如今进了盒。
        
        \newpage
        
        \section{40 无题}
        \begin{center}
            雷绍夫
        \end{center}
        
        猴山芭蕉绿暗,盒里苹果红甘,众猴侧目欲啖。
        
        雷猴疯癫病残,竟为电子扣饭,又遭四有群嘲,羞愧逃入井盖。
        
        \section{40 老猴与绍武}
        \begin{center}
            勒卉更
        \end{center}
        
        昔日深山老林中,老猴倒悬老山中,哪知老猴屁股重,噗通!老猴掉进池子中。
        
        今日有猴名绍武,猴老却说真理无,众人皆嘲老猴舞,呜呼!老猴一急进盒中。
        
        \section{42 丑奴儿·反雷}
        \begin{center}
            RobL61
        \end{center}
        
        少年参加八二六,不学无术。不学无术,文理不通便写著。
        
        晚年纠缠反雷派,看不清楚。看不清楚,表演猴戏被人辱。
        
        \newpage
        
        \section{43 浣溪沙·劝雷绍武一则}
        \begin{center}
            RobL61
        \end{center}
        
        满脑空想谬理堆,偏执疯狂令人悲,境地悲惨该怪谁。
        
        十年以来狂犬吠,百年之后无人会,科妄雷猴快改悔。
        
        \section{44 朝天子·咏雷}
        \begin{center}
            雷绍夫
        \end{center}
        
        盒前,群间,狂痴老猴现。造喷水器欲飞天,赤手摸零线。荒论连篇,无知尽显,似泼妇街间。看猿,叹猿,自轻何其贱\footnote{自轻:指晚节不保。}。
        
        \section{45 打油诗·观猴}
        \begin{center}
            勒卉更
        \end{center}
        
        有猴名叫雷绍武,终日把那邪说讲。邪说讲,把那邪说当真讲。
        
        众人看猴哈哈笑,皆把邪说当猴语。当猴语,猴子依旧不停语。
        
        众人为观猴,给猴喂香蕉。猴子一高兴,咏雷一篇出。
        
        众人觉有趣,皆学咏雷猴。猴子更高兴,咏雷百篇出。
        
        倘若众人弃猴去,老猴无蕉自苦恼。呜啊一声脑溢血,不久便要进盒中!
        
        \newpage
        
        \section{46 寻书词}
        \begin{center}
            亚瑟摩根
        \end{center}
        
        \begin{center}
            崇文院里寻墨槐,惯入深处人不猜。
        
            无意带将猴入怀,竟挑绍武出馆来。
        \end{center}
        
        \section{47 雷绍武赞}
        \begin{center}
            雷绍夫
        \end{center}
        
        \begin{center}
            无知咏雷空虚至极
            
            无德狡辩恐惧至极
            
            无耻辱骂泼妇骂街
            
            无赖雷猴无耻至极
        \end{center}
        
        \section{48 无题}
        \begin{center}
            雷华丰
        \end{center}
        
        \begin{center}
            大佛之下岷江边,弱论无知出妄言。
       
            志士求真天理胜,雷犬吠日世人怜。
            
            绍真寻理正道险,武备文修官学坚。
        
            尽力十年终败北,何必当初空钻研。
        \end{center}
        
        \newpage
        
        \section{49 千本雷}
        \begin{center}
            RobL61
        \end{center}
        
        \begin{center}
            雷绍武把d来约,
            
            雷绍武手握零线,
            
            雷绍武潜心钻研,
            
            运动力学。
            
            ~\\
            反对他的是脑残,
            
            心底无私天地宽。
            
            吧友把猴来看看,
            
            满嘴扯淡。
            
            ~\\
            雷猴绍武,满嘴荒唐言。
            
            咏雷藏字,我看不见。
            
            荒言谬论,坚持十年。
            
            任何反对都听不见。
            
            ~\\
            别人拆穿漏洞,我假装不在线。
            
            一提到八二六,我差点气急眼。
            
            不刊论闹笑话,谢谢正义之言。
            
            不会洋文还被麦瑟尔夫骗。
            
            ~\\
            羽扇纶巾孔明,公瑾不会打扮。
            
            金无赤足倒装,无知无赖纠缠。
            
            继续狗的狂吠,我素质没得看。
            
            肚子疼沃兹基看不太明白。
            
            ~\\
            我早年身世坎坷,
            
            有幸遇见杨正贤。
            
            回忆往昔的扣饭,
            
            省七毛钱。
            
            ~\\
            一碗水馒头两瓣,
            
            我的生活很简单。
            
            琴棋书画都精湛,
            
            怎会不堪?
            
            ~\\
            涌雷反雷,看猴戏表演。
            
            没人真正信我扯淡。
            
            隔着屏幕,侃侃而谈。
            
            可惜没有人认真看。
            
            ~\\
            别人拆穿漏洞,我假装不在线。
            
            一提到八二六,我差点气急眼。
            
            不刊论闹笑话,谢谢正义之言。
            
            不会洋文还被麦瑟尔夫骗。
            
            ~\\
            羽扇纶巾孔明,公瑾不会打扮。
            
            金无赤足倒装,无知无赖纠缠。
            
            继续狗的狂吠,我素质没得看。
            
            肚子疼沃兹基看不太明白。
            
            ~\\
            无可药救,雷绍武太憨。
            
            意淫瞎想,颅内实验。
            
            反雷电子,费心力劝。
            
            怎奈雷绍武自作贱。
            
            ~\\
            别人拆穿漏洞,我假装不在线。
            
            一提到八二六,我差点气急眼。
            
            不刊论闹笑话,谢谢正义之言。
            
            不会洋文还被麦瑟尔夫骗。
            
            ~\\
            肮脏龌龊一生,井盖下差点残。
            
            四川大学白读,我汉语都不善。
            
            晚年各种笑话,我全视而不见。
            
            一梦全能猴戏永存百万年!
        \end{center}
        
        \newpage
        
        \section{50 雷猴的绝命歌}
        \begin{center}
            雷绍夫
        \end{center}
        
        \begin{center}
            站在变电间
            
            斜眼笑
            
            被迫去证明
            
            杀人的雨从天上降临
            
            盒里的丧钟
            
            突兀的轰鸣
            
            诉说着$L=kmv$
            
            ~\\
        \end{center}
        
        \section{51 雷才叹}
        \begin{center}
            勒卉更
        \end{center}
        
        \begin{center}
            天边云野淡,灯下白发翁。
            
            题笔书佞学,不耻话肮脏。
            
            可怜年少时,属得好文章。
            
            而今耄耋年,老身众人嘲。
        \end{center}
        
        \newpage
        
        \section{52 反雷}
        \begin{center}
            雷招武
        \end{center}
        
        \begin{center}
            胸无点墨腹中空,自诩真理在心中。
        
            空谈运动需得力,妄谈床动人不动。
        
            万物构成皆电子,串联并联结构同。
        
            笑看邪说因何产,缘是雷猴跳出笼。
        \end{center}
        
        \section{53 反雷·井研老猴之歌}
        \begin{center}
            \Large \kaishu
            改编自雷绍武乐山大佛之歌
            \songti \large
            
            雷绍夫
        \end{center}
        
        \begin{center}
            天下的美景在四川,
            
            四川的美景在乐山。
            
            乐山的美景在雷宅,
            
            老狗和雷猴上下蹿,
            
            老狗呀雷猴呀上下蹿。
            
            猴亦是老狗啊,狗也是雷猴,
            
            脚踏飞机水,头顶大捆烟;
            
            徒手摸零线,身燃黑火焰,
            
            身燃黑火焰,断气进盒间。
        \end{center}
        
        \section{54 英语反雷一首}
        \begin{center}
            RobL61
        \end{center}
        
        \begin{center}
        Motion force theory is nothing.
        
        Anyone can prove this thing.
        
        ~\\
        No one do believe,
        
        I hope you will relieve.
        
        ~\\
        Take your time,
        
        to do something more meaningful.
        
        ~\\
        Do not trust your "supporters".
        
        They are great liars.
        
        ~\\
        Till the time went over,
        
        Till ignorant cranks disappear.
        
        ~\\
        No one would remember Mr. Ray
        
        and under joy.
        
        ~\\
        We will celebrate the day
        
        you and your theory are buried.
        \end{center}
        
        \newpage
        
        \section{55 卜算子}
        \begin{center}
            RobL61
        \end{center}
        
        \begin{center}
            平地一声雷,乐山一犬吠。自诩天才大智慧,智商太可悲。
            
            武斗早年黑,谬论晚节毁。十年谩骂尚未颓,绍武何其醉。 
        \end{center}
        
        \section{56 渔家傲}
        \begin{center}
            RobL61
        \end{center}
        
        运动力学全无据,荒谬学说难以续。乐山绍武心不死,猴难遇,无德无知但唏嘘。
        
        藏字咏雷奉宝玉,捧至天高知结局。至诚劝解当儿戏,太糊涂,明辨是非绍武需。
        
        \section{57 Ten things I hate about you}
        \begin{center}
            雷绍夫
        \end{center}
        
        I hate your tricks
        
        I hate your poems
        
        I hate your role as a clown
        
        I hate your absurd theories
        
        and your prejudice
        
        I hate your past
        
        I hate all 826 crimes you committed
        
        I hate your ill-tempered curses
        
        I hate you playing dead, refusing to apologize when you can't explain your ideas
        
        And most importantly, I hate you because you abandon your humanity, and deprave from man to monkey
        
        \section{58 十四行·雷}
        \begin{center}
            雷绍夫
        \end{center}
        
        \begin{center}
        推开盒子
        
        雷把谬论
        
        放入井盖
        
        ~\\
        它静静坐在
        
        猴山上
        
        雷展现自己无知无赖的地方
        
        ~\\
        一切都和管科的理论一样
        
        一切都反对
        
        太差太差的颅内实验
        
        \newpage
        
        ~\\
        雷仿佛
        
        一个移动的笑话
        
        一个愚昧执拗的老不死
        
        可他的一切遭遇都源于他自己的无德无知
        \end{center}
        
        
        \section{59 沁园春·反雷}
        \begin{center}
            拆家大主教
        \end{center}
        
        一碗烟火,四方时事,莽莽红尘。逾人间百年,黯黯思忖;雷氏谬论,心乱纷纷。星海横流,岁月成碑,反雷电子能几人?顾海内,诘碌碌雷粉,何异鸡豚?
        
        苟延残喘尚存,悲绍武无智头发昏。看桥头坠石,砰砰打脸;乐山杯土,臭不可闻。宇宙揭秘,狗屁不通,雷猴头上扣屎盆。立山头,笑雷猴绍武,都是儿孙!
        
        \section{60 寻雷老不遇}
        \begin{center}
            雷绍夫
        \end{center}
        \begin{center}
            盒外问电子,言雷闭关思。
            
            只在井盖里,不知生与死。
        \end{center}
        
        \newpage
        
        \section{61 十谬理}
        \begin{center}
            集体创作
        \end{center}
        
        \noindent\textbf{RobL61:}
        
        一谬理,零分之一等于一,何等荒谬愚蠢理?小学都懂,绍武不及,笑苹果原理!
        
        \noindent\textbf{RobL61:}
        
        二谬理,自命不凡运动力,打伞但为防雨滴。绍武不知,生活常识,科妄无人及!
        
        \noindent\textbf{雷绍夫:}
        
        三谬理,万物没有坐标系,都在独特位置里。白痴绍武,愚蠢至极,早日进盒里。
        
        \noindent\textbf{雷招武:}
        
        四谬理,电磁粒子构万物,并联串联NS极,冥顽不化,自诩真理,无知又无礼。
        
        \noindent\textbf{雷绍夫:}
        
        五谬理,恒星是发光物体,蜡烛点亮太阳系。天体物理,绍武不识,真无知至极。
        
        \noindent\textbf{RobL61:}
        
        六谬理,无知喷水式飞机,皆源荒唐运动力。川剧变脸,不认歪理,我真看不起!
        
        \noindent\textbf{RobL61:}
        
        七谬理,民科谬论水平低,绍武只会颅中臆。无知无德,无节无礼,都在看猴戏。
        
        \noindent\textbf{Mono6:}
        
        八谬理,相互作用不是力,不懂什么是定义。物理不佳,语文也差,枉上中文系!
        
        \noindent\textbf{雷绍夫:}
        
        九谬理,两球重者先落地,雷力等于KMV。没有基础,自创雷力,可笑又可气。
        
        \noindent\textbf{雷招武:}
        
        十谬理,羽扇纶巾非周瑜,不刊之论为贬义。年少无知,参八二六,真不精学艺。
        
        
        \section{62 相见欢}
        \begin{center}
            RobL61
        \end{center}
        
        乐山井研老猴,好吹牛,自比牛顿理论万年留。
        
        八二六,辩网友,糗事多,笑看无知民科绍武秀。
        
        \section{63 西江月·评雷绍武的826本性}
        \begin{center}
            RobL61
        \end{center}
        
        早年不堪回忆,绍武奉为珍馐。年少参加八二六,工厂井盖出丑。
        
        七十载人生漫,十余年都缩头。恶劣秉性不愿修,可悲顽固老朽。
        
        \newpage
        
        \section{64 菩萨蛮·重力加速度}
        \begin{center}
            RobL61
        \end{center}
        
        嘉州城下岷江流,愚翁绍武把石投。问掉了没有?坚信肆点玖。
        
        观录制视频,难让人取信。删除该视频,就当没事情。
        
        \section{65 天净沙·反雷(二)}
        \begin{center}
            RobL61
        \end{center}
        
        \begin{center}
        胡编谬理匆匆,脑袋朦胧空空,自诩惊雷隆隆。
        
        颅内臆想,对线反雷汹汹。
        \end{center}
        
        \section{66 满江红·平地惊雷}
        \begin{center}
            RobL61
        \end{center}
        
        平地惊雷,颅内臆,不胜怀悲。望嘉州,名仕辈出,奈何有雷!十年来光阴虚度,七十载白头无为。谈谬论,怎见老恩师?何其卑。
        
        自吹擂,无人慧。八二六,前途毁。运动力,沉浸歪理图一醉。可笑咏雷一千篇,愚翁智商被其摧。待老朽,收拾浆糊脑,莫为贼。
        
        \newpage
        
        \section{67 小改《咏雷·回“赵明毅”》}
        \begin{center}
            雷专打魔怔人
        \end{center}
        
        \begin{center}
            乐山雷氏承邪气,邪说妄想撼真理。
            
            今有勇士披甲征,化名唤做赵明毅。
            
            有知有德有秉性,有勇有谋有胆气。
            
            揪出雷氏邪徒日,辱骂之言俱还击!
            
            待到真理全胜日,扫尽一切民科敌!
        \end{center}
        ~\\
        
        \section{68 沁园春}
        \begin{center}
            雷华丰
        \end{center}
        
        地心邪说,地平谬论,作古百年。望科学圣土,万人竞往;真理大道,一马平川。乐山老朽,腹中空空,颅内实验出妄言。运动力,算公转周期,五百万年。
        
        除法大道至简,分苹果原理来计算。问公式公理,朝定夕改。雷言雷语,自命不凡。力电光热,数理化生。颅内一闪全推翻。吠不止,如丧家之犬,终成笑谈。
        
        \newpage
        
        \section{69 沁园春·对雷华丰词}
        \begin{center}
            RobL61
        \end{center}
        
        地心邪说,地平谬论,毒害百年。望全球寰宇,民智大开;科学高峰,无人不攀。乐山老朽,颅内出谬,自比智者歪理谈。运动力,希腊旧理论,无人稀罕。
        抄袭旧论也罢,竟无知无德把天贪。问拉格朗日,看不清楚。牛顿爱因,不屑去览。奥妙数论,哥德巴赫。大道至简天地宽。太无知,此井底之蛙,令人心寒。
        
        \section{70 反雷·改桃花源记}
        \begin{center}
            勒卉更\footnote{注:此文章原型桃花源记,魔改、大改。如有人名重合,纯属巧合!}
        \end{center}
        
        数十年前,有猴子妄言为业,胡言乱语于网络。有一日,忽逢贴吧,吧中众人,怡然自乐,便发歪理邪说于吧内。吧中众人,见猴子,乃大惊,问所从来。俱不答,以太差太差呼之。吧友甚异之,建雷吧,把猴墙之,便邀入吧。
        
        吧友闻有此猴,咸来闻讯。猴自云发现管科误错,设雷氏力学开创大道,以破牛顿焉。吧友问之,乃发现初中物理尚未及,无论高、大。此猴一一为所问而困,便言:不和太差太差的人争论。吧友见之,皆笑而不语。
        
        吧友又各复发帖而询问,猴皆答之,答不上来,则曰:太差太差,空恐泼无,人渣表现。众人皆逗之,雷吧遂大。
        
        雷吧遂大,忽有蛆来,吧友语蛆云:不足为外人道也。蛆即出,便扶向路,处处道之。蛆来,猴自封,遂离,不复得归。
        
        乐山徐士侠,高尚士也,闻猴同在乐山,欣然规往。未果,猴进盒。后遂无问津者。
        
        \section{71 如梦令·反雷三首}
        \begin{center}
        RobL61
        
        ~\\
        \Large \kaishu
        (一)
        \songti \large
        \end{center}
        
        沉浸谬理如故,从来不知悔悟。祭奠杨正贤,茫然无功所述。无语,无语,十年光阴虚度。
        
        \begin{center}
        ~\\
        \Large \kaishu
        (二)
        \songti \large
        \end{center}
        
        少年跳井逃命,晚年歪理不兴。面对反雷派,立刻现出原形。不幸,不幸,古稀依旧未醒。
        
        \begin{center}
        ~\\
        \Large \kaishu
        (三)
        \songti \large
        \end{center}
        
        谬论出自颅中,咏反皆为观众。十年一场空,何苦自造囚笼。如梦,如梦,此生一事无成。
        
        \newpage
        
        \section{72 杂诗·反雷}
        \begin{center}
            RobL61
        \end{center}
        
        \begin{center}
            自诩绍续新科学,无知老猴欲捞月。
            
            民科弱,难再续,智商绍武所必须。
            
            真理脍炙遍五岳,膜拜之风已完结。
            
            即已零零新纪年,献此诗为绍武谏。
        \end{center}
        
        \section{73 成语大全}
        \begin{center}
            Mono6\footnote{整理自 @辩证物理学 的贴子,有改动。}
        \end{center}
        
        雷老师研究师心自用,雷老师实验处心积虑。
        
        雷老师的成果无所不至,雷老师的功绩罄竹难书。
        
        对他的反对都是不刊之论,对他的质疑全都差强人意。
        
        雷老师理论被收弹冠相庆,雷老师理论被拒如丧考妣。
        
        \section{74 临江仙·二劝雷绍武}
        \begin{center}
            RobL61
        \end{center}
        
        岷江东逝浪淘尽,老狙脑中汹涌,意淫管科何其恐。吹擂真英雄,涌反皆观众。
        
        落日夕阳晚霞暗,沽名钓誉自颂,无名无誉真狗熊。劝君淡名利,遇蕉需从容。
        
        \newpage
        
        \section{75 Graze the Thunder}
        \begin{center}
            \textit{To the tune of “Graze the Roof (In-Game)” by Laura Shigihara (Plants Vs. Zombies OST)}
            
            Mono6\footnote{01:16和03:30两段经RobL61小改。}
        \end{center}
        
        \begin{center}
            [00:03]
            
            雷绍武\ 运动力
            
            不懂装懂约掉d
            
            人从床上惊坐起
            
            动的是床不是你
            
            ~\\
            
            [00:11]
            
            川大生\ 中文系
            
            不懂什么是定义
            
            不刊论差强人意
            
            看了觉得没问题
            
            问题
            
            ~\\
            
            [00:35]
            
            摸零线\ 没有电
            
            发电切割磁粒线
            
            南北极的串并联
            
            哪来什么化学键
            
            ~\\
            
            [00:43]
            
            地球绕\ 太阳转
            
            五百万年转一圈
            
            完全错误的表现
            
            愧对恩师杨正贤
            
            正贤
            
            ~\\
            
            [00:55]
            
            遇到问题我就说看不清楚
            
            不能自圆其说就说物体错误
            
            四点九加速度
            
            有人处理数据我就把视频删除
            
            攻击雷论你就是四无
            
            ~\\
            
            [01:16]
            
            雷吧已成猴园\ 四处埋伏危险
            
            多听吧主的意见
            
            人生已到暮年\ 多多注意晚节
            
            别再去丢人现眼
            
            年轻不学无术去参加革命
            
            老年荒唐理论却自诩天命
            
            涌雷反雷皆把你当做小丑
            
            外吧吧友也把你称为雷猴
            
            ~\\
            
            [01:47]
            
            雷绍武\ 运动力
            
            不懂装懂约掉d
            
            人从床上惊坐起
            
            动的是床不是你
            
            ~\\
            
            [01:55]
            
            川大生\ 中文系
            
            不懂什么是定义
            
            不刊论差强人意
            
            看了觉得没问题
            
            问题
            
            ~\\
            
            [02:19]
            
            摸零线\ 没有电
            
            发电切割磁粒线
            
            南北极的串并联
            
            哪来什么化学键
            
            ~\\
            
            [02:17]
            
            地球绕\ 太阳转
            
            五百万年转一圈
            
            完全错误的表现
            
            愧对恩师杨正贤
            
            正贤
            
            ~\\
            
            [02:39]
            
            当年欢迎有理有据来质疑
            
            十年过去智商素质逐渐降低
            
            反对都是无知
            
            合理讨论指出错误永远听不进
            
            四无人员竟是你自己
            
            ~\\
            
            [03:00]
            
            雷吧已成猴园\ 四处埋伏危险
            
            多听吧主的意见
            
            人生已到暮年\ 多多注意晚节
            
            别再去丢人现眼
            
            年轻不学无术去参加革命
            
            老年荒唐理论却自诩天命
            
            涌雷反雷皆把你当做小丑
            
            外吧吧友也把你称为雷猴
        \end{center}

            \newpage

        \poemwithst{76 沁园春}{小改咏雷版}{RobL61} %有副标题,使用\poemwithst
        
        \lid
        
        民科邪教,彼时嚣张,今且难逃。
        
        看神州上下,民智大开。绍武小丑,但供讥嘲。
        
        脑中邪说,口中谬论,心怀粪土却自豪。
        
        更有甚,实偏执疯狂,自比天高。 
        ~\\
        
        理论毫无成就,只能臆管科正飘摇。
        
        看早时无知,尚有礼节。近年辱骂,疯如魔妖。
        
        世间科学,万千理论,唯有雷理无人瞧。
        
        心狭隘,看千篇咏雷,自我高潮。
        
        \lidend
        ~\\
        
        \noindent 注:此为根据ID君意见小改的版本,原作如下:

        \kaishu
        民科邪教,彼时嚣张,今且难逃。看神州上下,民智大开。绍武小丑,但供耻笑。脑中邪说,口中谬论,心怀粪土却自豪。更有甚,实偏执疯狂,自比天高。 
        
        理论毫无成就,只能臆管科正飘摇。看早时无知,尚有礼节。近年辱骂,自以为傲。世间科学,万千理论,唯有雷理无人瞧。心狭隘,看千篇咏雷,自我高潮。
        \songti
        
    \newpage %根据需要,直接开始新的一页
        
    \poem{77 咒雷}{勒卉更} %没有副标题,使用\poem
                       \begin{center}
                       雷宅风景坏,日日伪科海。

                       老骨不识抬,终终唱戏来。 

                       后有反声起,智商如蝼蚁。 

                       早晚抱佞理,进入盒子里。
                       \end{center}
        
        %注:用\footnote插入脚注
     %强制换行符,用这个符号可以实现多个空行
        
    \poem{78 打油诗·抓猴}{勒卉更} %没有副标题,使用\poem
        
        \begin{center} %居中排版
        四川有乐山,山上有只猴。

        猴子太顽皮,天天来叫喊。

        大家来努力,把猴抓一起。

        抓来放哪里?放到盒子里。

        \end{center}
    
    \poem{79 定风波·三劝雷绍武}{RobL61}

    绵绵细雨春意浓,凄美时节酒一盅,再醉不及老猴勇。愚勇,蚍蜉撼树无人崇。 

    风卷残云落叶休,莫急,跳梁小丑自会怂。民科邪教无人从,劝雷,早日从真不卖弄。 
    
    \newpage
    
    \poem{80 水调歌头·反雷}{赵明毅}
    
     \lid
       万数谬论创,四海恶名扬。
       
       无端捏造,论述皆是臭而长。
       
       两种人才不懂,四有者全支持,梦里造天纲。
       
       坚持十余载,不化是雷郎! 
       ~\\

       肉电子,高温下,转成光。
       
       颅中实验,臆测歪理愈猖狂!
       
       劝汝回头是岸,不致晚节不保,专注写文章。
       
       顽固何须问?理解要智商!
       
    \lidend
      
      ~\\
      
    \poem{81 诉衷情·反雷}{RobL61}
    \lidlid
    
        当年自贱成老猴,求人把蕉投。
        
        人智断于何处?井盖下缩头。 
        ~\\
        
        盒未进,智先流,鬓已秋。
        
        人生如梦,猴戏十载,梦断嘉州。
    
    \lidend
        
    \newpage

    \poem{82 反雷}{雷绍夫}
     \setlength\parindent{6em}
     
         乐山之君子,井研之\textbf{雷\footnote{本诗歌集中,使用粗体、下划线等标记提示藏字。}}公。
         
         八二六时\textbf{候},洁身而自好,不与恶徒共。
         
         俯身三十载,屈尊下井盖,\textbf{为}保真理在。 
         
         后又出山,据理斥管科,有如\textbf{老}猴跃盒中。
         
         \textbf{不}畏无德人,教化电子众。
         
         \textbf{尊}为雷老狗,力学传千秋。
         
         知识\textbf{无}边界,诸学皆精通。
         
         管科\textbf{知}其论,无人不震悚 。
         
         牛爱若能闻,\textbf{无}地以自容。
         
         \textbf{德}高而望重,雷电于长空。

     \lidend

\poem{83 一剪梅·五劝雷绍武}{RobL61}
\begin{center}
    科妄少智太无脑,颅涌高潮,心狂自傲。 

    谬理空谈创邪教,自比天高,出丑见笑。 

    劝君心沉莫急躁,勤学思考,皆不可少。 
    
    夜郎自大勿再效,少盼香蕉,多学正道。


    \end{center}

\newpage
\poem{84 江城子·四劝雷绍武}{RobL61}
    
    疯者狂人事无成,以史鉴,皆可证。遥想独逸\footnote{独逸:即德国。},“大梦”被人憎。草菅人命败战争,山河碎,后世讽。 
    
    复劝绍武莫要疯,谬理论,无人奉。自比天高,科妄空想登。谨记恩师教诲赠,勤学习,白头等。
    ~\\

\poem{85 永遇乐·反雷}{乐山的雷公}

    千古江山,歪理无觅,雷绍武处。经典力学,全盘否定,妄言运动力。百万误差,忽略不计,地球转动绕柱。看电子,正负南北,串联并联成组。
    
    钻研十载,歪理邪说,独行民科长路。思而不学,劝而不听,年老愈发顽固。回首当年,不学无术,年少入八二六。凭谁问,绍武老矣,能诺奖否。

\newpage

\poem{86 渔家傲·反雷}{赵明毅}

    万千理论皆狗屁,老雷无耻弄猴戏,十载坚持从未腻。脑子里,不知屎尿何容器?
    
    自恃清高视天地,迷之自信忽非议,可笑至极仍蒙蔽。黄土里,棺材一口为栖地。
    ~\\

\poem{87 采桑子·反雷}{赵明毅}

    嘉州处处风景好,独有一糟。独有一糟:雷狗邪说叫喊高。 
    
    传播谬论批真理,放弃贞操。放弃贞操,猴戏民科终遁逃。
    ~\\

\poem{88 浣溪沙·反雷}{赵明毅}

    学界和平数百年,一只小丑跳出天。摆出歪理舞翩翩。 
    
    无耻宣扬歪谬论,自大忽视好心言。重归正轨亦何难?
    ~\\

\newpage

\poem{89 临江仙·反雷}{赵明毅}

    不知信众皆观戏,以为自悟天机。可怜年至古来稀,智商低下未能及。
    
    身况渐微何可怜,内心昏聩低迷。人寰撒手首神离,只添玩笑供人提。

\poem{90 讨咏雷派檄文}{雷专打魔怔人}
\begin{center}
    芸芸众生,失理则泠;蒸蒸灵息,无仪则獍。
    
    人有温热,天有暗明;荣于善行,耻于恶名。

    乐山雷氏,不仁劣性;世居岷江,无德无情!

    当思官科至理,一统真理盛兴。

    拭目今日,民科横行;

    道不能畅,理不得清;

    上无所申,下入窘境。

    不守节操,造作成性。

    食人之不敢食,饮人所未尝饮,

    弃德之昭明,抛心之魂灵,

    阳为人之形状,阴则阿谀后庭。

    登堂妒忌,吧友争相沽名;

    入室纷争,以讨雷氏高兴;

    谗语掩杀,真理怏怏无命。

    怂恿污流,演义肮脏之风;

    邪说万能,可以不要魂灵。

    无知,卖笑纵情;

    无德,抛却亲情;

    无耻,丧尽人性;

    无赖,远离友情。

    如今神人共怒,天地更所不容。

    犹有暗藏祸心,妄想天崩。

    以达乱中取胜,饱囊别宫。

    科学之蝥虫,祸害已成;

    乐山之贼子,何其求成?

    噫嘻!反雷幡然悟醒,

    独尊官科至圣。

    柳丝飘拂,觉春来之绿生;

    荷珠映彩,喜夏花之幽净。

    愚一介布衣,无所适名。

    谨奉宗旨,循真理之志,演反雷之风。

    气冲云天,立志长缨!

    顺乎人心,爰举义旗!

    吁呼之诚,竭尽所能!

    但求清风,呼唤真理,呐喊人性!

    复苏复苏兮,真理!回归回归兮,人性!

\newpage

\end{center}
\poem{91 雷猴理论来源}{析境}
\begin{center}
    篡改事实,混淆是非;颠倒黑白,不辨真假。
    
    无理无据,无知无赖;完全错误,太差太差。

\end{center}

\poem{92 猴道难}{雷公助我}

    噫吁嘻,雷猴叫哉!猴道之难,难于进盒雷\footnote{进盒雷:指雷老师探寻真理,进入真理的宝盒。}!运动\footnote{运动:这里指雷老师的运动力理论。}与电子,雷力何茫然!尔来七十又几岁,不与管科通人烟\footnote{通人烟:指说人话。}。民科绍武有猴道,可以四无又疯癫。
    
有知有德壮士生,然后正确中出\footnote{这里“中出”存在歧义,有人说“支持”也可。}相钩连。上有运动速度\footnote{运动速度:运动的一定是物体的速度,这里是符合理论的。}之高标,下有SNNS之回川。管科之技尚不得过,猿猱下床床攀援\footnote{猿猱下床床攀援:猴子从床上攀援而下,而实际是床从猴子上攀援而上。}。川渝何盘盘,川大之才不中干。缩头进井仰胁息,恩师正贤坐长叹。

 问君地球何时还?五〇二七九〇三。只见雷进八二六,捂头乱窜绕井间。又闻扣饭心不甘,书记难。猴道之难,难于上青天,使人听之没脸见。
 
全吧反串新天地,雷公一梦做上天,发帖对线争喧豗,看不清楚实验雷。其脑也如此,嗟尔逗猴之人胡为乎来哉?

 运动力学绍武开,一夫当关,万夫备盖\footnote{盖:代指上文“盒”,即真理的宝盒。}。所守之理论,化为火与柴。朝避吧友,夕编咏雷,磨牙吮血,杀人如麻。贴吧虽云乐,不如早还家。猴道之难,难于真理添!楼主西望长咨嗟!

\poem{93 青玉案·反雷}{赵明毅}

    七十人语\footnote{语:yù}从心顾,仅雷老,行歪路。人世百年何短苦?怜君未有,利他之处,唯显无为碌。
    
    星辰日月流如顾,但耻一生陷深误!日堕黄昏蒙万物。若求良谥,行将就木,及早歧途住!

\poem{94 蝶恋花·六劝雷绍武}{RobL61}

    武斗大败井盖跑,举止荒唐,狼狈太可笑。四年虚度学习少,净行不义何处逃? 
    
    年高体衰心未老,十年表演,何苦设囚牢?劝君莫要恋香蕉,当以人脑换猴脑。

\newpage
\poem{95 江城子·反雷 }{赵明毅}

绍武何曾解科学,智商缺,海内绝。爬至零线,猴屁股长撅。现象证明雷论错,不听劝,意坚决。 
    
杨公应恨教此学,智常竭,把d约。只盼回头,能把谬真抉,如是经年入土后,可含笑,保操节。

\poem{96 卜算子·反雷}{赵明毅}

少时战斗团,老矣传歪理。历史汪洋如微石,未作涟漪起。 
    
如梦前半生,似幻终年里。腐朽残身含笑终,感叹无悲喜。

\poem{97 水调歌头·一评雷绍武 }{RobL61}

    狂人在何处?乐山井研有。笃信颅内大梦,自贱成老猴。诚心相劝不顾,冷嘲热讽取辱,疯狂令人呕。狂犬尚吠日,处世如老狗。 
    
智渐衰,言益愚,让人忧。不学无术,怎见恩师于身后?少年作恶武斗,晚节自毁心魔,何苦要自囚?枉度七十载,哀乐盒中奏。

\newpage
\poem{98 风入松·二评雷绍武}{RobL61}

    无知老朽常扮猴,晨起拭眼眸。电子火花到处透,零线无流,无可药救。但笑牛顿无知,意淫官科无谋。
    
地球公转万年久,反对皆荒谬。沽名钓誉成小丑,狂吠似狗,乐上心头。早年荒废学业,晚年频频上钩。

\poem{99 怒雷}{勒卉更 }

    千古悠悠,顽莫如,雷绍武。蔑视真理唱劣戏,假大师,真无知! 
    
傻论频出仍不耻,乐山桥上实验试。无知无耻至如此。可笑!何胆还作咏雷诗!

\poem{100 钗头凤·七劝雷绍武}{RobL61}

    人情险,世情恶,谬理无人认真谈。不听劝,但心寒。愈发偏执,颅内空研。叹,叹,叹!
    
心正端,少臆想,收拾朽脑向前看。多纳言,莫等闲。迷途知返,师欣九泉。返,返,返!

\newpage
\poemwithst{101 南乡子·八劝雷绍武 }{改编自震惊百里的南乡子·咏雷 }{RobL61}

    名利何处求? 但来雷吧仙府游。 一发谬论百人呼。 吹牛, 沽名钓誉实可忧。 
    
弃名少上钩, 安度晚年过伞寿。 何苦自贱成老猴? 心愁, 劝君退隐褪污垢。

~\\
\poem{102 钗头凤·n讽雷绍武}{小野寺麗}

    伏生碌,囊萤馥,义生贤士功德铸。雷人悟,愚言入,几多讹误,谬答如注。怒,怒,怒。
    
年迟暮,邪心固,一生时日空虚度。黄泉路,风潇簌,视其尤库,竟将毒\footnote{毒:指826}助。覆,覆,覆。

\newpage
\poem{103 扇泼猴}{乐山的雷公}       

    岷江东去,乐山四顾,世间谬理在何处。看雷公,谈理数,胸无点墨仍顽固,歪理邪说如粪土。蠢,雷绍武,猴,雷绍武。

\poem{104 扇泼猴·山坡羊咏雷其二十二改}{乐山的雷公}

    懒惰如猪,无力似鼠,乐山老朽性迂腐。似八婆,似怨妇,安敢把那贤能妒,真理面前似若侏儒!愚,雷绍武,蠢,雷绍武!

\poemwithst{105 扇泼猴}{改自山坡羊·一劝赵明毅}{乐山的雷公}

    楚歌四面,孤城紧闭,民科小丑势已去。谬误出,歪理提,负隅顽抗不量力,一败涂地已成定局!劝,早日拥真理,诫,早日拥真理!

\newpage
\poem{106 扇泼猴}{Thunder Monkey}

\setlength\parindent{11em}

    \textbf{四}年谬误,
    
    \textbf{川}中硕鼠,
    
    \textbf{井}下曾为栖身处。
    
    \textbf{研}万物,只在颅,
    
    \textbf{雷}理不传心中苦,
    
    \textbf{犬}牙乱咬恰似泼妇。
    
    \textbf{狂},雷绍武!
    
    \textbf{吠},雷绍武!

\poem{107 扇泼猴}{Thunder Monkey}

    \textbf{乐}土天府, 
    
    \textbf{山}水如故, 

    \textbf{雷}动风涌官学固。 

    \textbf{犬}声出,心中苦, 

    \textbf{终}其一生难明悟。 

    \textbf{成}事不足谁与为伍,

    \textbf{奸},雷绍武; 

    \textbf{佞},雷绍武!

\lidend

\chapter{赵明毅《山坡羊·反雷》系列}

\begin{center}
    (《反雷集》序号108\textasciitilde134)
\end{center}

\subsection{山坡羊·反雷系列总序:}
  
\kaishu
面对无耻无赖的涌雷电子“乐山的雷公”的山坡羊·咏雷,我,赵明毅在此与之斗曲,全写山坡羊,至今更败一筹,特将这些篇目发表出来。同时,如果雷绍武本人看到了的话,希望你多看看《山坡羊·反雷其十三》,这是我的真心话,希望你迷途知返,这样以后你不多的岁月结束后也能有个好点的名声。
    ~\\

\songti

\poem{108 山坡羊·反雷其一}{赵明毅}

    盒中有物,原为绍武,张牙舞爪面容怖。智如猪,性轻浮,三千歪理皆发布,曾有哪篇无谬误处?愚,雷绍武;蠢,雷绍武!

\newpage

\poem{109 山坡羊·反雷其二}{赵明毅}

    雷猴拥簇,迷人注目,邪说每读欲呕吐。惜智无,望前途,伤心年至古稀处,活在梦幻自作论主。腐,雷绍武;朽,雷绍武!
~\\

\poem{110 山坡羊·反雷其三}{赵明毅}

    贫屋独处,盒中漫步,反雷电子无穷数!风云起,天地怒,民科无智行尽路,邪说背道终为灰土。邪,雷绍武;恶,雷绍武!
~\\

\poem{111 山坡羊·反雷其四}{赵明毅}

    阴魂无属,坚持如故,雷猴坏水填满腹。罪当诛,灭恶徒,猴戏再耍年四五,可笑民科即将作古。猖,雷绍武;狂,雷绍武!

\newpage

\poem{112 山坡羊·反雷其五}{赵明毅}

    不懂算数,头脑似木,学识短浅令捧腹。骂他徒\footnote{他徒:其他人},是科奴,若在其脑细细顾,闹虫跑狗飞鸟行鼠。狡,雷绍武;诈,雷绍武!
~\\

\poem{113 山坡羊·反雷其六}{赵明毅}

    真理天斧,劈裂邪物,决斩雷猴如毙兔。民科徒,势已孤,首领已行将就木,一朝树倒猢狲跑路。灭,雷绍武;除,雷绍武!
~\\

\poem{114 山坡羊·反雷其七}{赵明毅}

    神统万古,科学终普,造福苍生幸福路。民科猪,绍武出,反对科学骂学术,井底之蛙蚍蜉撼树。渺,雷绍武;小,雷绍武!

\newpage

\poem{115 山坡羊·反雷其八}{赵明毅}

    不知何故,打翻陈醋,面对真学甚嫉妒。欲荼毒,著邪书,谬论皆入绍武肚,妄图撼动科学做主。蛮,雷绍武;横,雷绍武!
~\\

\poem{116 山坡羊·反雷其九}{赵明毅}
    
    德行如鼠,猖狂如虎,天下皆望尸肉腐!凭栏哭,意踌躇:民科本属无耻物,何时进盒化为白骨?!贪,雷绍武;婪,雷绍武!
~\\

\poem{117 山坡羊·反雷其十}{赵明毅}

    不信科普,傻论无数,百劝执迷仍不悟。论若输,便言辱,无理取闹众人怒,手捧谬论仍试修补。顽,雷绍武;固,雷绍武!

\newpage

\poem{118 山坡羊·反雷其十一}{赵明毅}

    不知万物,不识一数,学者面前弄大斧。自傲殊,自信笃,秒遭打脸如吃醋,酸苦急躁出言侮辱。恼,雷绍武;羞,雷绍武!
~\\

\poem{119 山坡羊·反雷其十二}{赵明毅}

    思维迂腐,无耻如鼠,小丑妄把科学阻。名声污,智不足,众叛亲离不醒悟,怀怨井蛙终将入土。孤,雷绍武;独,雷绍武!
~\\

\poem{120 山坡羊·反雷其十三}{ 赵明毅}

    行将就木,仍行歪路,渐行渐远未停步。明诗书,知今古,只因无知有此故,弃明投暗走入迷处。痛,雷绍武;惜,雷绍武!

\newpage

\poem{121 山坡羊·反雷其十四}{ 赵明毅}

    不堪细读,不认谬误,雷氏歪理马脚露。咏雷徒,势已孤,历史茫茫如尘土,一切民科定入棺木。卑,雷绍武;微,雷绍武!
~\\

\poem{122 山坡羊·反雷其十五}{ 赵明毅}

    地球转速,百万计数,滑稽理论使捧腹。零线出,被零除,前人理论以万数,皆被民科碰瓷围堵。放,雷绍武;肆,雷绍武!
~\\

\poem{123 山坡羊·反雷其十六}{赵明毅}

    真理如虎,天震人怒,科学卫军凌江渡!民科除,大道复!正论之枪所到处,邪说信徒奔命似鼠!狼,雷绍武;狈,雷绍武!

\newpage

\poem{124 山坡羊·反雷其十七}{赵明毅}

    自称朴素,实则朽腐,陈旧观念心底住。欲高呼:“回正途!”反被自大言侮辱,无奈只等猴自作古。迷,雷绍武;惑,雷绍武!
~\\

\poem{125 山坡羊·反雷其十八}{赵明毅}

    数典忘祖,出言相辱,何能将汝谬论补?似狗扑,狂叫呼,毫无知识作基础,雷猴撒泼直至入土。疯,雷绍武;癫,雷绍武!
~\\

\poem{126 山坡羊·反雷其十九}{赵明毅}

    劝言不入,坚持如故,装死无能发狂怒。写歪书,难通读,数物化文皆涉处,万科全无知识基础。昏,雷绍武;聩,雷绍武!

\newpage

\poem{127 山坡羊·反雷其二十}{赵明毅}

    生于巴蜀,养于天府,浪费栽培入歧路。问苍天,此何如?人类败类是此~徒,不肖子孙民科硕鼠!羞,雷绍武;耻,雷绍武!
~\\
    
\poem{128 山坡羊·反雷其二十一}{赵明毅}

    若有出入,物体错误,多次修改雷常数。视频出,不相符,便说时间秒秒顾,如墙头草随风簌簌。软,雷绍武;弱,雷绍武!
~\\
    
\poem{129 山坡羊·反雷其二十二}{赵明毅}

    自望天府,愁水满腹:忠言绍武耳不入。怜情抒,泪沾服:古稀人生因此输,死抓谬论不肯降服。悲,雷绍武;哀,雷绍武!

\newpage
    
\poem{130 山坡羊·反雷其二十三}{赵明毅}

    故人已去,空留雷鼠,众叛亲离众人辱。入迷途,友难阻,人生岔路一再误,何其不惜感人肺腑?遗,雷绍武;落,雷绍武!
~\\
    
\poem{131 山坡羊·反雷其二十四}{赵明毅}

    不惠绍武,尽惹人怒,且听大军铿锵步!真理出,科信徒,自成卫队捍正主,科学浪潮涤净巴蜀!斩,雷绍武;决,雷绍武!
~\\
    
\poem{132 山坡羊·反雷其二十五}{赵明毅}

    硕鼠硕鼠,无食我黍!万人共愤民科属。雷猴徒,欲反扑,怎敌真知科学住,灰飞烟灭化为尘土!诛,雷绍武;杀,雷绍武!

\newpage
    
\poem{133 山坡羊·反雷其二十六}{赵明毅}

    人生短苦,少有命主,百年能成大旗鼓。然雷徒,十年扑,竟成此多歪论著,无限笑料流传万古!耻,雷绍武;辱,雷绍武!
    
\poem{134 山坡羊·反雷其二十七}{赵明毅}
    \subsection{题记:}
    \normalsize
    \kaishu

    若涌雷电子按照承诺把他的《山坡羊·咏雷其二十六》作为封笔之作,这将会是山坡羊这一组的曲的最后一篇。

    前二十六首的最后两句“雷绍武”前一个字连在一起组成26个词,罗列在此,这将是反雷电子声势浩大地举起血红的科学之旗的序幕!

    \songti

    \begin{center}

    \noindent 看绍武,\textbf{顽固耻辱恼羞卑微放肆狼狈迷惑疯癫昏聩},
    
    何其\textbf{孤独},使人\textbf{痛惜悲哀},\textbf{遗落}于人世。

    \noindent 笑雷猴,\textbf{愚蠢软弱腐朽邪恶猖狂狡诈渺小蛮横贪婪},
    
    不知\textbf{羞耻},理当\textbf{灭除诛杀},\textbf{斩决}当何时?

    \end{center}

    \subsection{正文:}
    \large
    千军拥护,万人声促,绍武进盒当从速!雷学徒,尽皆诛,大道真理如震怒,科论万宗终将光复!奔,雷绍武;亡,雷绍武!
    
    \newpage

    \kaishu
    另附诗一首,以彰对山坡羊斗诗中,反雷电子收官之作的完结:(此诗编入反雷·170)

    \songti
    \begin{center}
        \normalsize
        黄田坝前白日昏,千重骇浪犹腾奔。

        绍武幼时争战地,往来种作和气存。

        浩劫当年事堪叹,绍武试把天地撼。

        老来重拾翻覆志,谬论层出妄开天。

        乐山冻饥民科惨,理论不进渐成殇。

        老矣偏执不听劝,愁容满面临岷江。

        争论落败所自致,贴吧奔窜如亡羊。

        堆床十斛仅麦屑,一勺入口无米汤。

        当时狂论意何取,离离满目悲禾黍。

        终有一日化白骨,黄土棺中朽体腐。

        若有人忆耍猴事,或见陆续来吊古。

        傻论笑料终消散,鲸波蚀尽战场土。
    \end{center}

    \large
    ~\\

    \hfill{(此系列完)}

\chapter{}

\poem{135 扇泼猴·其一}{RobL61}

脑中空空,臆想汹汹,胡编谬理太匆匆。雷隆隆,思淙淙,沽名钓誉心彤彤,邪说尽出绍武口中。灭,民科从。斫,民科从。

\poem{136 扇泼猴·其二}{RobL61}

骄躁老猴,血沥心呕,但为谬理常奔走。躯已朽,几时休?真理大厦尚未否,小丑民科终成老狗。卑,民科猴;微,民科猴!

\poem{137 扇泼猴·其三}{RobL61}

惛怓老猴,言语不周,猴言猴语使人逗。八二六,井盖投,年至古稀智商忧,下场凄惨自取咎由。灭,民科猴!绝,民科猴!

\newpage

\poem{138 扇泼猴·其四}{RobL61}

见香蕉投,不顾衰朽,费尽心机太鄙陋。人劝汝,不思咎,反成狂犬咬人手,为人所殴方井盖投。贱,民科猴!悲,民科猴!
~\\

\poem{139 扇泼猴·其五}{RobL61}

疯癫绍武,言语恶毒,臆想管科已被除。人皆奴,唯尔悟,自诩学者知识无,民科小丑但供凌辱。骄,雷绍武;躁,雷绍武!
~\\

\poem{140 扇泼猴·其六}{RobL61}

绍武跋扈,癔症反复,傻脑当中谬理驻。劝不顾,把蕉护,自设囚笼盒里住,畜言狂叫恩师不孚。疯,雷绍武;癫,雷绍武!

\newpage

\poem{141 扇泼猴·其七}{RobL61}

自设桎梏,晚节不顾,遇见香蕉癫病复。望不孚,傻如故,一生光阴便虚度,可怜民科尚未悔悟。猴,雷绍武;戏,雷绍武!
~\\

\poem{142 扇泼猴}{Mono826}

无耻狂吠,无能狂怒,民科绍武如泼妇。德也无,知也无,今日放肆传谬误,明天就要往盒里住!悲,雷绍武;惨,雷绍武!
~\\

\poem{143 扇泼猴}{Mono826}

克服险阻,开辟通途,扫尽一切民科奴!邪说除,正理普,喷水飞机约d术,与你一并作了古!消,雷绍武;除,雷绍武!

\chapter{反雷对联集体创作}
\begin{center}
    (《反雷集》序号144\textasciitilde167)
\end{center}

\subsection{序:}

\kaishu
反雷对联的创作,是由Mono6(特别声明:与上述Mono826无关)发起,编辑部全员参与,并广泛吸收大群群友意见的一项创作活动。对联的创作相对于诗词可能较简单,但文学性未必亚于诗词,读来亦朗朗上口。热烈欢迎大家参与到反雷对联的创作中来。
\songti

\section{144 反雷对联(其一)}

\shanglian{Mono6}{投机取巧,误入八二六}
\xialian{Mono6}{兴妖作怪,枉活七十七}
\xialian{狗力大仙}{沽名钓誉,枉活七十七}

\section{145 反雷对联(其二)}

\shanglian{RobL61}{劝解不听,年老昏聩扮猴戏}
\xialian{雷绍夫}{真理未识,暮时残朽学犬鸣}

\section{146 反雷对联(其三)}

\shanglian{Mono6}{潜藏井底,武斗中拾得性命}
\xialian{雷绍夫}{暗出盒外,文争里丢掉脸皮}

\section{147 反雷对联(其四)}

\shanglian{RobL61}{藏字不识,大脑中全是歪理}
\xialian{Mono6}{咏雷尽录,思想里皆为邪说}

\section{148 反雷对联(其五)}

\shanglian{Mono6}{无知无德,运动力无人认可}
\xialian{雷绍夫}{有礼有节,反雷理有据依存}

\section{149 反雷对联(其六)}

\shanglian{乐山的雷公}{胸无点墨,泼猴竟敢研数理}
\xialian{雷绍夫}{腹缺诗书,老狗安能究诗文}

\section{150 反雷对联(其七)}

\shanglian{小野寺麗}{雷猿昏心聩智终成遗老}
\xialian{RobL61}{朽木败智无知后为疯猴}

\section{151 反雷对联(其八)}
\shanglian{雷绍夫}{峨眉山下,乐山水前,有猿猴通人言,\\ 泼妇骂街,狺狺狂吠,妄称其能究物理}
\xialian{乐山的雷公}{天府国中,大佛像边,有蜀犬能吠日,\\ 不学无术,日日胡言,竟云其将大道昭}

\section{152 反雷对联(其九)}
\shanglian{RobL61}{瞧瞧雷公:少智甚矣,科妄甚矣,狂犬吠日,可悲甚矣}
\xialian{Mono6,引绍武名句}{看看楼主:空虚之极,恐惧之极,泼妇骂街,无耻之极}

\section{153 反雷对联(其十)}
\shanglian{小野寺麗}{今朝猿陷埔墁招人摈}
\xialian{雷绍夫}{明日犬落井盖惹群嘲}

\section{154 反雷对联(其十一)}

\shanglian{小野寺麗}{今日犬彘匣中荡}
\xialian{Mono6}{明天猿猱线上行}

\section{155 反雷对联(其十二)}
\shanglian{小野寺麗}{十年沉浮,运力谬论吧间绱袖}
\xialian{Mono6}{一夜风波,南航笑谈群内流传}

\section{156 反雷对联(其十三)}
\shanglian{RobL61}{看无知猿猴摸线嚎啕啸}
\xialian{乐山的雷公}{观愚蠢老翁进盒呜咽啼}

\section{157 反雷对联(其十四)}
\shanglian{Уоgцгт}{猴起如厕厕动猴不动}
\xialian{小野寺麗}{犬伏进盒盒移犬不移}
~\\

\section{158 反雷对联(其十五)}
\shanglian{Mono6}{欲证加速四点九,老朽桥上扔石头, \\ 逐帧分析打脸,把视频一删,忙曰否}
\xialian{RobL61}{曾言公转五百万,泼猴群中扯慌言, \\ 几图佐证扇面\footnote{扇面:与“打脸”同义。},将傻论三申,自食言}

\newpage

\section{159 反雷对联(其十六)}
\shanglian{雷绍夫}{八二六老猴犯贱,民科谬论害人不浅,几时进盒摸零线}
\xialian{RobL61}{四点九傻论心寒,臆想邪说令吾感叹,今朝雷犬愈发憨}
\xialian{Mono6}{七十七愚翁发狂,歪理邪说出口无穷,明日入土败万年}
~\\

\section{160 反雷对联(其十七)}
\shanglian{小野寺麗}{雷于雨田进井盖}
\xialian{雷绍夫}{盒为合皿吓老猴}

\newpage

\section{161 反雷对联(其十八)}
\shanglian{RobL61}{零分之一,二货绍武,三年雷吧,加速四点玖; \\ 五(无)脑雷猴,六百藏头,七百暗讽,可悲八二六; \\ 自取九(咎)由,十年漫扮猴}

\xialian{Mono6}{十年也九(久),八方歪理,七旬老狗,辩论六神糊; \\ 五兆周期,四处否认,三番抵赖,无知二百五; \\ 不可一世,零线晚节输}
~\\

\section{162 反雷对联(其十九)}

\shanglian{Mono6}{赵公大德,真理将胜}
\xialian{RobL61,引经典藏字}{雷神弱智,民科必亡}

\newpage

\section{163 反雷对联(其二十)}

\shanglian{Mono6}{咏雷千余首,多少钓鱼辱骂}
\xialian{RobL61}{邪说逾百篇,大都颅臆谬谈}
\xialian{乐山的雷公}{藏字数百条,几分谩骂讥嘲}
~\\

\section{164 反雷对联(其二十一)}

\shanglian{RobL61}{不懂装懂大狂犬,笑孔明羽扇纶巾,真无可药救}
\xialian{乐山的雷公}{人云亦云老泼猴,嘲反雷不刊之论,绝贻笑大方}

\newpage

\section{165 反雷对联(其二十二)}

\shanglian{Mono6}{几只蠢猪,几条疯狗,不足为惧}
\xialian{雷公助我、Mono6}{一个老朽,一介愚夫,无脸而谈}
\xialian{街角,咏雷}{一碗开水,一个馒头,亦可成才}
~\\

\section{166 反雷对联(其二十三)}

\shanglian{Mono6}{动荡年间,唯有绍武猴入井}
\xialian{小野寺麗}{太平盛世,竟出为民牛弹琴}

\newpage

\section{167 反雷对联(其二十四)}

\shanglian{雷绍夫}{绍武妄言,恰似一条疯狗}
\xialian{雷绍夫}{反雷高论,宛如几朵奇葩}
\shanglian{雷绍夫、Mono6}{咏猴妄言,恰似两三疯狗}
\xialian{雷绍夫、Mono6}{反雷高论,宛如千万明灯}
~\\

\hfill{(此系列完)}

\chapter{}

\section{168 定风波}
        \begin{center}
            雷公助我
        \end{center}
        
        莫听吧友无德声,不惧管科真理渗。自我催眠无需劝。谁怕?川渝乐山一愚翁。
        
        成键电子运动力,荒诞!正贤恩师摇头叹。回首半生当猴看,进盒,不知悔悟不知返。
        ~\\
        
        \section{169 满江红·九劝雷绍武}
        \begin{center}
            RobL61
        \end{center}
        
        十年也久,望雷吧,已成猴园。看绍武,本为常人,何苦自贱?千古悠悠出真论,十年匆匆谈妄言。劝雷老,忘了空名誉,莫狂研。
        
        忆往昔,念正贤。七十载,人生艰。至耄耋,劝君淡泊成贤。幻灭麻痹千咏雷,长驱鬼魅忘虚愿。既白头,念晚节为重,听此言!

        \newpage
        
        \section{170 七古·反雷}
        
        \begin{center}
            赵明毅
        \end{center}
        
        \kaishu
        本诗亦作为赵明毅二十七首山坡羊·反雷,即反雷108 \textasciitilde 反雷134的总结。
        \songti

        \begin{center}
        黄田坝前白日昏,千重骇浪犹腾奔。
        
        绍武幼时争战地,往来种作和气存。
        
        浩劫当年事堪叹,绍武试把天地撼。
        
        老来重拾翻覆志,谬论层出妄开天。
        
        乐山冻饥民科惨,理论不进已成殇。
        
        老矣偏执不听劝,愁容满面临岷江。
        
        争论落败所自致,贴吧奔窜如亡羊。
        
        十斛堆床仅麦屑,一勺入口无米汤。
        
        当时狂论意何取,离离满目悲禾黍。
        
        终有一日化白骨,黄土棺中朽体腐。
        
        若有人忆耍猴事,或见陆续来吊古。
        
        傻论笑料终消散,鲸波蚀尽战场土。
        \end{center}

        \newpage

        \section{171 满江红·反雷}
        \begin{center}
            赵明毅
        \end{center}
        
        狂妄民科,握歪理、手持零线。自以为,呼风唤雨,策雷驱电。无所不为传谬论,行将就木终天谴。待几年,白骨入灵棺,泥中烂。  
        
        忠言进,常觉反;诚告汝,应听劝!负杨恩师诲,九泉羞见。一梦全能终是梦,几回能够听人谏?莫等闲、趁尚未西归,迷途返!
        ~\\

        \section{172 扇泼猴}
        \begin{center}
            乐山的雷公
        \end{center}
        
        胆小如鼠,无力如兔,古稀老朽不识数。目光浅,学识疏,安敢把那真理妒,胸无点墨属四无!愚,雷绍武,钝,雷绍武!  
        ~\\
        
        \section{173 扇泼猴(改自《山坡羊·二劝赵明毅》)}
        \begin{center}
            乐山的雷公
        \end{center}
        
        风云四起,寒烟满地,民科者无立足地。真理明,大道立,困兽犹斗谈何易,科学圣论万年不移!愚,雷氏理,蠢,雷氏理!

        \newpage
        
        \section{174 七律}
        \begin{center}
            RobL61
        \end{center}

        \begin{center}
        辛苦作词逾九篇,欲谏绍武弃谬研。
        
        苦口婆心尽珠玑,冗词赘句皆胡言。
        
        沉浸猴戏全不知,沽名钓誉尚汗颜。
        
        一梦全能图一醉,盒中宇宙百万年。
        \end{center}

        \section{175 江南春}
        \begin{center}
            平常有事
        \end{center}

        \begin{center}
        愚不醒,教难明。孤身行谬论,年老至高龄。
        
        无知胜似猿猴蠢,时日无多思莫明。
        \end{center}

        \section{176 钗头凤}
        \begin{center}
            平常有事
        \end{center}
        
        成雷力,遭人斥。乐山雷狗何时毕。雷猴恶,人人觉。猿猴愁绪,欲人休驳。错。错。错。    
        
        春重入,人无术。智贫人弱西时日。科知薄,功何卓。人形仍在,口言粗浊。龊。龊。龊。
        \section{177 菩萨蛮·一生愚}
        \begin{center}
            小野寺麗
        \end{center}
        
        凌云壮志出师毕,双流文苑研学济。不想斗争齐,风雷革命移。追思藏井史,绍武多羞耻。诞论脑中积,无节今沫滴。

        \section{178 盒我所欲也}
        \begin{center}
            Mono6
        \end{center}
        
        盒,我所欲也;井盖,亦我所欲也。二者不可得兼,舍盒而取井盖者也。晚节,亦我所欲也;香蕉,亦我所欲也。二者不可得兼,舍晚节而取香蕉者也。晚节亦我所欲,所欲有甚于晚节者,故不为苟得也;四无亦我所恶,所恶有甚于四无者,故钓鱼有所不辟也。如使人之所欲莫甚于晚节,则凡可以得晚节者何不用也?使人之所恶莫甚于四无者,则凡可以辟钓鱼者何不为也?由是则晚节而有不保也,由是则可以辟鱼钩而上钩也。是故所欲有甚于晚节者,所恶有甚于四无者。非独绍武有是心也,猴皆有之,绍武能十年如一日耳。

        一开水,一馒头,得之则生,弗得则死。呼尔而与之,两种人弗受;蹴尔而与吾,吾亦欣然受之也。晚节则不用香蕉而受之,晚节于我何加焉!为名誉之美,家庭之和,贴吧之网友得我与?乡为武斗而不受,今为名誉之美为之;乡为装死而不受,今为家庭之和为之;乡为空恐泼无而不受,今为贴吧之网友得我而为之;是亦不可以已乎?此之谓失其本心。

        \section{179 天净沙·反雷(其三)}
        \begin{center}
            RobL61
        \end{center}
        
        \begin{center}
        默默窥屏潜藏,暗暗踢人肮脏,悻悻离群感伤。
        
        不胜悲怆,八二六甚难防。

        ~\\
        \end{center}

        \section{180 七绝}
        \begin{center}
            RobL61
        \end{center}

        \begin{center}
        猴言犬吠曝群间,无德小人射暗箭。
        
        今朝咬骨不松口,余生永为民科犬!

        ~\\
        \end{center}

        \section{181 点绛唇·反雷}
        \begin{center}
            平常有事
        \end{center}
        
        无耻雷猴,不求真理唯留误。脑遗章序,难有存知处。
    
        终日征途,仍是无安度。应维誉,为能回顾。却把香蕉嗅。

        \newpage

        \section{182 反雷}
        \begin{center}
            人不动
        \end{center}

        \begin{center}
        天上星辰数万年。三十七载为一天。
        
        行运周转总有数,五百万回是一环。

        零火火零零无电,水滴落地如刀剑。
        
        铁鸟喷水行云际,老骥伏枥寿千年。

        世上万物俱有数,唯有笑话数不完!

        ~\\
        \end{center}

        \section{183 将进盒}
        \begin{center}
            人不动
        \end{center}
        
        \begin{center}
        雷不见,官科谬论天上来,几经流转不复回。
        
        雷不见,虚度光阴悲白发,朝如才栋暮成悲。
        
        人生得意须进学,莫使无知空为才。
        
        天生蠢材难有用,光阴逝去不复来。

        琴棋书画且为乐,会须有智白鹤飞。
        
        师牛顿,伽利略,将进盒,笔莫停。
        \end{center}

        \newpage

        \section{184 定风波}
        \begin{center}
            雷绍夫
        \end{center}
        
        井盖当中十载游。乐山花落几重秋。乱语胡言狂已久。枯朽。形如痴傻老疯猴。
        
        忆昔当年八二六。禽兽。德知荒废知识帚。形单影只牛马走。粗陋。妄尊自大使人忧。
        ~\\

        \section{185 反雷}
        \begin{center}
            乐山的雷公
        \end{center}

        \begin{center}
        大雪纷飞云\textbf{雀}啼,

        民科盘踞\textbf{正}路迷。

        运动力学\textbf{全}然错,

        创新电子皆\textbf{顽}愚。

        弱冠不学出谬\textbf{论},

        古稀无术提歪\textbf{理}。
        
        铲除绍武光\textbf{世}道,
        
        昭明真理\textbf{雷}声稀。
        \end{center}
        
        \newpage

        \section{186 满庭芳·反雷}
        \begin{center}
            赵明毅
        \end{center}
        
        装死娴熟,时常诡辩,不知漏洞百出。以为明理,看百万愚夫;何谙自己被耍,演猴戏、引人驻足?苦悲道,何其难劝,反被骂科奴!
        
        观长存日月,永流江海,人世何足?似小丑,疯癫自大狂呼!未过百年身灭,何人忆,有是人乎?夕阳堕,雷猴将去,可笑亦孤独。
        ~\\

        \section{187 念奴娇·三评雷绍武}
        \begin{center}
            RobL61
        \end{center}
        
        零分之一,数不识,妄言结果为一。创新电子,南北极,高中化学不及。颅内臆想,嘴上空谈,激荡脑电子。咏雷千篇,多少暗语藏字。

        遥想革命当年,井盖下缩首,产军\footnote{产军:即产业军。}环伺。若逢难题,效那年,入盒回避装死。川渝学府,原登科高中\footnote{登科高中:即高考成绩优异。中,去声。},本为荆梓\footnote{荆梓:比喻人才。}。不学无术,老来无德无知。

        \newpage
        
        \section{188 菩萨蛮·反雷}
        \begin{center}
            赵明毅
        \end{center}
        
        黄田坝里惊天吼,刀枪棍棒寻敌首。一晃岁如梭,故人年已多。年高何有脑?吧内无理吵。年少古稀时,一生偏且执。
        ~\\
        
        \section{189 长相思·反雷}
        \begin{center}
            赵明毅
        \end{center}
        
        \begin{center}
        江水流,岷水流。垂老孤人满脸愁,阴云密雾忧。
        
        忆千仇,恨千仇。固守邪说何日休?自狂无理猴!
        
        ~\\
        \end{center}

        \section{190 长相思·劝雷}
        \begin{center}
            平常有事
        \end{center}

        \begin{center}
        盒中囚,井中囚。无耻雷猴满面愁,驳言环四周。
        
        春不休,冬不休。征战何时有敛收,人应知耻羞。
        \end{center}

        \newpage

        \section{191 八声甘州·有感}
        \begin{center}
            普露普露
        \end{center}
        
         纵横民吧数个春秋,一载入寒冬。恶云翻涌起,凄风冷雨,黄叶枯松。电子若离困苦,盒中无电休。唯两岸猿猴,不知忧愁。

         难书彦宏罪咎,临绍武贴吧,唯有蛆多。忆昔时吧景,欢乐无穷多。现如今、Q群讲学,传真理、何处权威多?请看猴,钻井盖下,傻论邪说!
         ~\\

        \section{192 七律·看雷绍武最新神论有感}
        \begin{center}
            RobL61
        \end{center}
        
        \begin{center}
        乐山井研凌云西,老雷钻研运动力。
        
        放电倒装太阳晒,火光飞机翔天碧。
        
        沙漠电子创新论,基因分子荡涟漪\footnote{基因分子荡涟漪:雷绍武认为基因分子是DNA的一部分,DNA中包含水分子。}。
        
        蜀犬吠日贻人笑,老朽傻论难以继。
        \end{center}

        \newpage

        \poemwithst{193 反雷·Like a barking dog}{改编自\textit{Like a rolling stone} (by Bob Dylan)}{RobL61}
        
        Once upon a time you seemed so fine
        
        Never admit your 826 crime, didn't you?
        
        People call, say ``Beware old man, you're bound to fall"
        
        You thought they were all abusin' you
        
        You used to laugh about
        
        Official scienctists that was hangin' out
        
        Now you still bark so loud
        
        Now you still seem so proud
        
        Never seems to be worryin' about your own box
        

        How does it feel?
        
        How does it feel?
        
        To be a crank ``oh no"
        
        Like a complete unknown?
        
        Like a barkin' dog?
        
        
        Aw, prince of the truth and mr. never wrong.
        
        They're all laughin' thinkin' of your joke.
        
        Writing all metaphors for you
        
        You'd better aware
        
        You're the best amuse
        
        You have your brian, but you never use
        
        When you face yonglei, ya can't refuse
        
        Your fictional dream, will never be due
        
        Your fake glory, when you in box, will be gone.
        

        How does it feel?
        
        Aw, how does it feel?
        
        To a crank oh no
        
        With no direction and hope
        
        Like a complete unknown
        
        Like a barkin' dog.
        ~\\

        \section{194 虞美人·反雷}
        \begin{center}
            赵明毅
        \end{center}
       
        \setlength\parindent{6em}
        四川春雨如期至,一载寻常始。
        
        老雷年岁近八十,满脑邪说难劝似呆痴。 
        ~\\

        百遭落败应觉耻,在世留何事?
        
        改邪归正受真知,趁未西归尽早破顽执!
        \lidend

        \newpage

        \poemwithst{195 四季——雷猴忏悔录}{小改陈奕迅同名歌曲}{乐山的雷公}

        \begin{center}
        我记起那年春天
        
        得我一个期刊的欺骗
        
        和奖状的伪造 浪费走了钱
        
        我记起那年暑天
        
        地球又在经过近日点 冷热
        
        离太阳近远
        
        最后已事过境迁 黄田坝风景已变
        
        再度回想那一天 川大哪位少年
        
        又再路过事发景点 痛伤少不免
        
        仍是会往井盖里钻
        
        尚记得当天产业军的刀剑
        
        我记起那年秋天
        
        独自屹立乐山桥头边
        
        丢石头做实验 把自己打脸
        
        我记起那年的冬天
        
        算出公转一圈五百万年
        
        只有误差一点点
        
        最后已事过境迁 乐山风景已变
        
        再度回想杨正贤 愧对恩师无面
        
        又再路过红薯良田 扣饭少不免
        
        仍是会归还叶子烟
        
        泪染进攻132那一天
        
        季度里事过境迁 顽猴终于蜕变
        
        再没留恋七角钱 往日两位少年
        
        就算让理论再钻研 会否不改变
        
        仍是被取笑 打脸
        
        回归光明正道感恩不怨
        
        我感激这年春天
        
        红薯开遍祝福相献
        
        待那硕果成长在田野间
        \end{center}

        \section{196 反雷(小改185)}
        \begin{center}
            乐山的雷公
        \end{center}
        
        \begin{center}
        大雪纷飞云\textbf{雀}啼,
        
        日薄西山\textbf{正}路迷。
        
        运动力学\textbf{全}然错,
        
        创新电子皆\textbf{顽}愚。
        
        误入歧途荒\textbf{学}术,
        
        冥顽不化\textbf{力}渐疲。
        
        破除民科\textbf{世}道复,
        
        天高云淡雨\textbf{雷}稀。
        \end{center}

        \section{197 反雷}
        \begin{center}
            Mono6
        \end{center}
        
        \begin{center}
        雷公八二六,才高井下秀。

        歪理闻华夏,无人出其右。

        ~\\
        \end{center}

        \section{198 反雷}
        \begin{center}
            Thunder Monkey
        \end{center}
        
        \begin{center}
        君住岷江头,颅内思宇宙。
        
        日日产真理,共与江水流。

        ~\\
        \end{center}

        \section{199 反雷}
        \begin{center}
            Thunder Monkey
        \end{center}
        
        \begin{center}
        四川有四川过,乐山可乐山居。
        
        井研\textbf{于井研理},千佛明千佛律。
        \end{center}

        \newpage

        \section{200 楚辞·反雷}
        \begin{center}
            雷绍夫
        \end{center}
        
        \begin{center}
        昔井研之雷公兮,无人智出其右。
        
        彼力学之泰斗兮,颅内可观宇宙。
        
        管科见之怆悢兮,牛爱闻之愈愁。
        
        瞬万变而位一兮,床动而人不动。
        
        ~\\
        静而若处子兮,动则如猿猱。
        
        光火而星恒兮,喷水飞机构造。
        
        彼大道其至简兮,无端约分玄奥。
        
        不顾身而传真理兮,摸零线进盒终老。
        \end{center}

        \section{201 雷歌}
        \begin{center}
            雷绍夫
        \end{center}
        
        雷兮雷兮,何异之有。蜡烛星火,盒中宇宙。运动力学,牛马奔走。夜半惊起,人静床动。

        雷兮雷兮,何奇之有。民科浩淼,毋出其右。理论雄奇,一发难收。四无但见,三缄其口。

        雷兮雷兮,何哀之有。本虽良才,自甘堕落。不学无术,为乱造祸。为老不尊,歪理邪说。

        雷兮雷兮,何愚之有。言语混乱,思维鄙陋。无知无德,辱骂群友。众皆恶之,赐名雷猴。

        \section{202 反雷}
        \begin{center}
            Mono6,街角小改
        \end{center}
        
        \begin{center}
        田中有深井,深井藏巨龙。
        
        巨龙产真理,真理入田中。
        \end{center}

        \section{203 猴说}
        \begin{center}
            平常有事
        \end{center}
        
        世有绍武,然后有井盖猴。井盖猴常有,而绍武不常有。故虽有顽猴,受辱于贴吧人之手,气死于盒子之间,不以井盖称也。
        
        猴之入井者,一食或甚三红薯。书记者不知其能入井而扣也。是猴也,虽有进井之能,食不饱,力不足,才美不外见,且欲与常猴等不可得,安求其能进井也?
        
        骂之不以其道,打之不能进其井,疯之而不能通其意,执策而临之,曰:“天下无猴!”呜呼!其真无猴邪?其真不知猴也!

        \section{204 渔家傲}
        \begin{center}
            Mono6
        \end{center}
        
        细雨润田滋川蜀,蜀中有叟名绍武。绍武颅中究万物。究万物,谬论连篇无穷数。
        
        蜀犬欲王令捧腹,腹无诗书学无术。无术只把官科妒。官科妒,枉活耋寿难瞑目。
        
        \section{205 江城子·评雷(其二)}
        \begin{center}
            Mono6
        \end{center}
        
        寒窗十载也曾读,览贤书,展宏图。怎料当年,取巧入歧途。遥想黄田争战处,不入井,命呜呼。
        
        光阴似箭青春除,井中居,何时出?傻论谁听,年暮甚孤独。辱骂纠缠装死术,如泼妇,胜四无!
        ~\\

        \section{206 三上}
        \begin{center}
            雷绍夫、Mono6
        \end{center}
        
        雷绍武虽生贫贱,而少有恶习。在群中时尝语群友,言平生惟好读书,坐则读科妄野史,卧则读群友赞歌,上厕则阅自产真理。盖未尝顷刻释卷也。
        
        大田鳟亦言:段贤香同在屎坑,每走厕必挟歪理以往,辱骂之声琅然,闻于远近,亦有知有德如此。雷公因谓田鳟曰:“余平生所产真理,多在‘三上',乃‘桥上’、‘床上’、‘厕上’\footnote{桥上,指乐山桥头扔石头实验;床上,指人不动床动理论,也讽刺雷绍武颅内实验白日做梦;厕上,不作解释。}也。盖惟此尤可以属思尔”。
        
        \newpage

        \section{207 三上}
        \begin{center}
            Thunder Monkey
        \end{center}
        
        雷公虽生长农家,而少信官学。在乐山时尝语信众,言平生惟好研理,坐车则思动静,卧则摸零线,上厕则推床。盖未尝顷刻弃理也。
        
        王甲鱼亦言:雷公在贴吧,入吧必挟真理以往,“正确”之声琅然,闻于远近,亦有知有德如此。雷公因谓甲鱼曰:“余平生所产真理,多在‘三上',乃‘桥上’、‘床上’、‘厕上’也。盖惟此尤可以属思尔”。
        \section{208 雷力行}
        \begin{center}
            Υυτ
        \end{center}
        
        官科昌,雷道亡,牛顿真理在中央。
        
        古往今来咸称口,灿若星辰官道上。
        
        雷氏小丑欲跳踉,哭声直上干云霄。
        ~\\

        贴吧学者问吧友,吧友但云运动力。
        
        或有一年五百万,数论要用苹果算。
        
        网上怒把吧友干,现实常往井盖钻。
        
        君不见:贴吧咏雷十百万,原是老头自己扮!
        ~\\

        纵有雷某打太极,雷氏力学无东西。
        
        咏雷电子无物理,只管哓哓复读机。
        
        反雷勇且正,雷某不吭声。譬如某个群,未停作怪音。
        
        群友问道勤,某人汗津津。信知理闻名,反倒作诗频。
        
        胡说八道圆不满,妄言妄语咏雷听。

        君不见,民科吧,一片谬论一眼假。
        
        新鬼烦冤旧鬼哭,天阴雨湿声啾啾。

        \section{209 清平乐·反雷}
        \begin{center}
            平常有事
        \end{center}
        
        前途幻杳,歪理仍需了。脑丧真知随处跳,而却并非年少。
        
        确如夕日残花,潇潇白鬓年华。只望此时回首,为求生后名佳。
        \section{210 四无}
        \begin{center}
            Mono6
        \end{center}
        
        \begin{center}
        无知妄言文章中,无德辱骂贴吧上。
        
        无耻装死井盖里,无赖翻脸不认账。
        \end{center}

        \section{211 四无}
        \begin{center}
            旋律蓝
        \end{center}
        
        \begin{center}
        无知公转百万年,无德贴吧出狂言。
        
        无耻躲向井盖边,无赖多次自否决。
        \end{center}

        \section{212 四有}
        \begin{center}
            玛奇玛
        \end{center}
        
        \begin{center}
        有知浑搅物理界,有德渴登功名殿。
        
        有礼四无当宝剑,有节作猴与人辩。

        ~\\
        \end{center}

        \section{213 四有}
        \begin{center}
            咏二列一大入
        \end{center}
        
        \begin{center}
        有知何谓运动力,有德何出暴脾气?
        
        有耻何来装失忆,有赖何故辱真理?

        ~\\
        \end{center}

        \section{214 反雷对联(其二十五)}
        \shanglian{Mono6}{惟蜀有才}
        \xialian{RobL61、Mono6}{于雷失智}

        \section{215 反雷对联(其二十六)}
        \shanglian{Mono6}{人人皆可放电}
        \xialian{Mono6}{个个都能打雷}

        \section{216 反雷对联(其二十七)}
        \shanglian{Mono6}{攻乎异端,谓人四无,只恐晚节不保}
        \xialian{RobL61}{斥之反动,称其五黑\footnote{五黑指黑五类,下联描述雷绍武826的恶行。},但惧前途无光}
        
        \section{217 反雷对联(其二十八)}
        \shanglian{Mono6}{怒斥吧友,失教少条,空度十一载}
        \xialian{Mono6}{空谈妄言,不学无术,枉活七十年}

        \section{218 反雷对联(其二十九)}
        \shanglian{雷公助我}{岷山乐山犬狂吠}
        \xialian{雷公助我}{零线火线盒发威}

        \section{219 反雷对联(其三十)}
        \shanglian{Mono6}{大田鳟进盒,普天同庆}
        \xialian{雷绍夫}{老雷猴出柜,举世齐欢}

        \section{220 反雷对联(其三十一)}
        \shanglian{集体创作}{雷绍武千古}
        \xialian{集体创作}{四川大学万岁}

        \section{221 反雷对联(其三十二)}
        \shanglian{Thunder Monkey}{雨打川地,大战黄田坝,雷氏谬论彷徨枯井上下}
        \xialian{Thunder Monkey}{禾丰天府,远望北斗星,科学真理叱咤学坛西东}

        \section{222 反雷对联(其三十三)}
        \shanglian{Mono6}{烂梗传千里,神蛆兔蛆抗蛆,魑魅魍魉,尽兴妖作怪}
        \xialian{小野寺麗}{谬论传百吧,雷猴段猴王猴,蚊蝇蜱蟑,皆欺世盗名}

        \poem{223 两老猴辩日}{雷绍夫}

        大田鳟出游,见两老猴辩日。一猴名雷绍武,一曰王金甲。雷猴言:日者,发光发热物体也。蜡烛即日,日即蜡烛。雷绍武力学泰斗,光焰万丈,亦太阳也。王猴无知无德,岂能识太阳?王猴恼,反唇相讥:日者,太空之中天体也。昔者有王为民星,撞击地球,遭致灾祸,王为民星即古太阳也。王金甲神游太虚,伟大至极,亦太阳也。雷绍武不识此理,岂非糠糟鄙里之徒?

        雷猴大怒,曰:人渣表现!空虚之极,恐惧之极,泼妇骂街,无耻之极!起而击王猴。王猴亦暴起。 

        大田鳟见曰:此二猴皆科妄。无知猪狗,不可与有知田鳟同语。速来学我永动机,来舔我脚!雷王二猴益怒,与田鳟混斗。须臾,三猴皆毙。
        ~\\

        \poem{224 生于井盖,死于匣内}{Mono6}

        饭扣于红薯之中,大志举于恩师之间\footnote{大志举于恩师之间:指在恩师杨正贤的激励下开始了“宇宙揭秘”的“研究”。},武斗举于川大之中,黄田坝匿于井,毕业证得于桥\footnote{毕业证得于桥:有传言称雷绍武曾在九眼桥伪造毕业证。},雷理论发于吧。故天将降大任于雷公也,必先苦其心志,劳其筋骨,饿其体肤,疯癫其脑,行拂乱其所为,所以成其四无,曾益其所不能。
        雷恒过,然后不改;困于心,衡于虑,而后遁\footnote{遁:逃避。这里指当自己理论出现问题并遭到他人质疑时,雷绍武假装下线不回答。};辱于吧,骂于群,而后喻\footnote{喻:被人知晓。这里指别人可以从雷绍武的辱骂行为中看出其四无本性。}。入则有藏字咏雷,出则尽“雷猴进盒”者\footnote{这里的入和出分别指雷绍武进群和雷绍武离开群时群里的表现。},雷恒亡。然后知生于井盖而死于匣内也。

        \newpage

        \poem{225 反雷·骚体}{渺天汉}
        \begin{center}
        巴蜀乐山之猿属兮,其皇考曰大圣。

        观其性顽而难教兮,锡其字号而污名。

        名之曰绍武兮,称之曰雷猴。

        天锡其书画修能兮,而专注于谬论。

        弃幽兰而刳朽木兮,鄙江离若粪壤。

        鸷不与之同飞兮,则孤高而自赏。

        汩日月之疾驰兮,然年岁之不其与。

        零落于尘烬兮,怀愤懑而长终。

        秽论之不绝兮,难继灵修之言。

        昔牛顿之真理兮,鄙之猖披尔窘步。

        忽奔走以明其知兮,反愈行其谗道。

        期迷行之欲反兮,荃不知吾辈中情。

        阻其黄昏而改路兮,不知好意而齌怒。

        后视其若猴兮,投之以香蕉。

        其喜洋洋而怡然兮,髣髴若得志。

        于是群起而戏兮,至此不知己为乐。

        嗟其何可笑兮,劝君莫仿其行。
        \end{center}

        \newpage

        \poem{226 贺新郎·反雷}{赵明毅}

        夜半千声静,郁梧桐、无言临月,水沉甘梦。孤苦何人独踱步,水映沧桑身影。凭栏想、不觉微冷。思至少时飘渺事,忆往昔、脑海如明镜。呓似喊,待谁应?

        犹记激荡风云景。与白林,共谈大志,投身革命。曾欲三山皆倾覆,坚信终将得胜。陷迷路,遁逃入井,狂乱十年忽作史,已白发,一世迷茫性。泪欲下,梦初醒。

        \poem{227 闻雷猴NS歪理有感}{雷绍夫}
        \begin{center}
            九眼桥外乐山城,NS排列组合分。

            雷公歪理真名世,千载谁堪伯仲间?
        \end{center}

        \poem{228 苏幕遮·反雷}{赵明毅}

        暮临山,空悲叹,谬论层出,只换旁人厌。日落人生光渐暗。惜尚执迷,思想何曾换? 
        
        忆儿时,风雨乱,地下藏身,懦弱全然显。真理卫军宏志展,腐朽民科,遁井活尸颤!

        \poem{229 七绝·观绍武空恐泼无(原诗)}{Mono6}
        \begin{center}
        脑中无物时时惧,心底空虚处处狭。
        
        老来泼骂真无耻,年少甘为井底蛙。

        ~\\
        \end{center}

        \poem{230 七律·观绍武空恐泼无}{赵明毅}
        \begin{center}
        脑中无物时时惧,心底空虚处处狭。
        
        老来泼骂真无耻,年少甘为井底蛙。
        
        谬论百错只装死,迷途渐远不识家。
        
        死后黄土渗朽体,扫入历史臭浮渣。

        ~\\
        \end{center}

        \poem{231 七绝·观绍武空恐泼无}{平常有事}
        \begin{center}
        年少无知井里惧,老来无耻心中狭。
        
        真知确理脑后放,此后只为井底蛙。
        \end{center}

        \newpage

        \poem{232 七绝·观绍武空恐泼无}{Mono6,ID君小改}
        \begin{center}
        颅头恐惧时时怅,心底空虚处处狭。
        
        年少甘为壁上草,此生沦作井中蛙。
        \end{center}

        \poem{233 南乡子·反雷(戏作)}{赵明毅}
        
        \kaishu
        戏作顶针小词,不合格律,但图耍雷一乐。
        
        \songti
        岷江横小舟,舟上一人并一猴。猴捧零线高挥手。手抖,抖入井盖乐悠悠。
        
        悠闲岸上走,走到井上看猴愁。愁己被电焦了头。头油,油焙叶子烟没收。

        \poem{234 反雷马赛曲}{RobL61}
        \begin{center}
        Allons défendeurs de la vérité.

        Le jour de gloire est arrivé 

        Entendez, ces animaux ignoble

        Moquent nos noble science.

        Moquent nos noble science.

        Regardez-vous, devant l'ordinateur.

        Mugir ces féroces gibbon.

        Il criere sa victoire.

        Mais il ne sait pas sa fin

        Debout! Pour la vérité.

        Debout! Pour la science.

        Marchon, marchon!

        Sans faux science, 

        Nous libérons le monde.
        
        ~\\
        我们走吧,真理的捍卫者们!
        
        光荣的一天来到了。
        
        听,那些无耻的禽兽,
        
        竟然嘲弄高贵的科学。
        
        竟然嘲弄高贵的科学。
        
        你可看见,在那电脑前,
        
        凶残的猿猴正在咆哮。
        
        他在呼号他的胜利,
        
        却不知,他的末日将临。
        
        起来!为了真理!
        
        起来!为了科学!
        
        前进,前进。
        
        没有伪科学,
        
        世界才会解放。
        \end{center}

        \poem{235 十六字令三首(其一)}{Mono6、雷绍夫}
        \begin{center}
        雷,自诩新声向旧吹。真愚昧,将入死人堆!

        绍,无知小丑房梁跳。每叫嚣,何人听你闹!
        
        武,德如犬彘识如鼠。势已孤,邪说将入土!

        ~\\
        \end{center}

        \poem{236 十六字令三首(其二)}{赵明毅}
        \begin{center}
        雷,千万拙诗自捧吹。邪说著,愚蠢少思维。

        雷,骄傲牛皮月下飞。行猴戏,膨胀使人悲。
        
        雷,莫再执迷陷不回!听人劝,苦海尚能归!

        ~\\
        \end{center}

        \poem{237 十六字令三首(其三)}{赵明毅}
        \begin{center}
        雷,虚妄如猴倒是非。黄田坝,井盖下颜灰。

        绍,猖狂谬论撒泼闹。耍赖多,围观群众笑。

        武,将临正理奔如鼠。小丑般,民科终作古。
        \end{center}

        \newpage

        \poem{238 十六字令三首(其四)}{虎爪吃布丁}
        \begin{center}
        雷,如厕无需己去回。人不动,夜半把床推。

        雷,实验无需器械随。观颅内,宇宙必循规。
        
        雷,答辩无需据是非。唯批判,尽显自心亏。

        \end{center}

        \poem{239 十六字令三首(其五)}{旋律蓝}
        \begin{center}
        雷,无人识得你是谁。雷学颓,徒把时光赔。

        绍,口出狂言如猿啸。跳脚叫,空得万人笑。

        武,胡言乱语如遗毒。看雷鼠,伪理终作古。

        \end{center}

        \poem{240 十六字令三首(其六)}{Mono6}
        \begin{center}
        \textbf{雷},\textbf{苟}入歧途甚可悲。\textbf{于}川大,\textbf{春}秋五度颓。

        \textbf{绍},\textbf{无}知一世使人笑。\textbf{自}高傲,\textbf{键}上瞎胡闹。

        \textbf{武},拒不\textbf{认}错遭群辱。\textbf{理}皆误,埋入\textbf{泥}土处。
        \end{center}

        \kaishu
        注:三首都有太好太好的内容(雷狗愚蠢、绍武自贱、无人理你),但为此内容和格律有所牺牲。
        \songti

        \poem{241 仿雷公旨}{雷绍夫}

        大雷猴盒里,八二六井盖里\ \ \ \ 雷老师雷猴的宣谕:涌雷电子每根底,反雷电子每根底,兀那投机电子每根底,往来贴吧水友那听着也么道。力学众理兴盛者,必雷猴行命的主宰有。天覆的,地载的,多生灵几有的不知有。札萨几有的有。凡雷猴说着的呵,你每理解支持者。群子里,没勾当的休入者,凡是拿住,便问拿住的人要者。凡雷猴的歪理呵,你每听着。凡雷猴的邪说呵,你每信者。反雷电子,有见呵歹的一般,也不怕那甚么!若是他每反雷者,揭穿雷力谬误者,我便井盖行向也么个! 
        ~\\
        
        \kaishu
        翻译:雷猴雷老师在井盖和盒里发号施令\ \ \ \ 涌雷电子,反雷电子,投机电子和贴吧水友们听着:雷氏力学理论兴盛,是因为雷猴我有知有德,主宰运动力。所有力都归雷猴我管,所有定理以雷猴我的颅内实验为准 雷猴的话,你只能理解支持,不许反对。四无人员不要进群发乱七八糟的东西,不然我就让群主和管理人员判断你是不是人渣。雷猴歪理再蠢你们也必须听,雷猴邪说再错你们也必须信。要是反雷电子无理取闹,无礼纠缠,也不怕雷猴我的辱骂封禁和屏蔽!但是,要是他们在雷猴我斥责了之后仍然不停下,继续坚持真理,让雷猴我回答不上问题,雷猴我就溜进井盖里装死了!
        \songti

        \newpage

        \poem{242 反雷}{斯捷潘阿琳}

        \begin{center}
        乐山有猴雷绍武,年过七七快入土。
        
        无知偏执出谬论,搏得笑名众人辱。
        \end{center}

        \poem{243 点绛唇·反雷}{赵明毅}

        歪理猖狂,十三春秋嚣张气。无人搭理,膨胀生乖戾。
        
        宋老威风,尿醒民科起。此场戏,扇猴一记,誓荡平天地!

        \poemwithst{244 水调歌头·反雷}{改氢氦锂铍硼咏雷}{Mono6}

        万数邪说创,四海臭名扬。无端捏造理论,妄言官科狂。先有微分约去,又创电磁粒子,颅内造文章。虚度十余载,可叹是雷郎! 

        雷群中,专家前,显仓皇。四无本性,众目之下方得彰!谬理无人赞许,论战败而装死,只在井中藏。犯错何须认?但言“要智商”!

        \newpage

        \poem{245 无题}{壹仨贰井盖居士}
        \begin{center}
        平地一声雷,惊醒无数人。

        为何猴不醒,缘是真无为。
        \end{center}

        \poem{246 无题}{壹仨贰井盖居士}
        \begin{center}
        一切真理全无用,只留颅内脑生锈。
        
        千般道理全不认,空余雷力不知羞。
        \end{center}

        \poem{247 闻雷老师大败有感而言}{壹仨贰井盖居士}
        \begin{center}
        千般道理尽是无用,十年雷力净是虚空,不学无术无用。
        
        七十七年有何建树,二十四载成何体统,只留一阵西风。
        
        ~\\

        遥想当年黄田坝下,正值意气风发,好似千军万马。

        转眼今日年老体衰,灯枯油尽烛蜡,何能挽回当下。
        
        ~\\

        我劝诸君莫学雷,一事无成使人悲。

        年少正值勤学时,金榜题名耀宗碑。
        \end{center}

        \poem{248 十六字令·咏宋}{赵明毅}
        \begin{center}
        宋,直言似剑真知诵,论证精,民科羞面重。
        
        天,通晓科学述万千。接连辩,谬论复无言。
        
        佑,宜将剩勇追穷寇。灭谬说,雷猴惊遁走。

        ~\\
        \end{center}

        \poem{249 藏字戏作一首}{赵明毅}
        \begin{center}
        \textbf{宋}老闲来辩民科,

        人言\textbf{天}意\underline{雷}将扫。
        
        众人\underline{绍}意\textbf{佑}真知,
        
        \underline{武}功文治齐呼\textbf{好}。
        
        雷猴\underline{脑}里\textbf{有}诡辩,
        
        但见\textbf{知}行\underline{无}德操。
        
        不识离子为何\underline{物},
        
        宛如小丑犹哓哓。
        \end{center}

        \newpage

        \poem{250 鹧鸪天·反雷}{赵明毅}

        期盼专家十年间,到时只见不能前。理说未胜耍无赖,诡辩逃脱闭不言。
        
        一巴掌,把猴扇,醒于梦里落神坛。回头一望人生路,尽付空虚化海澜。

        \poem{251 醉太平·贺雷猴论战惨败}{雷绍夫}

        官科论昭,雷公败逃。难应答语寥寥。以何言智高?

        理论乱抄,不知角标。热量单位为蕉。谬言堪耻笑。

        \poem{252 观雷宋论战有感二首}{Mono6}
        \begin{center}
        雷,雷,雷,电子邪说危。
        
        专家一驳斥,全都化成灰。
        
        ~\\
        宋,宋,宋,打得雷猴痛。
        
        真理巨斧前,绍武有何用!
        \end{center}

        \poem{253 咏雷·九六七}{玄幙}
        \begin{center}
        管无人,雷道昌!昭昭天理见何方?
        
        今有绍武抗权威,中寿拱木立冢上!
        
        一腔赤血为真义,正言敢逆愚劣腔。
        
        诚当共睐运动力,清明乾坤卒旌飏!
        \end{center}
        \kaishu

        注:此为名咏实反的典范之作。雷绍武不通其意,收入《咏雷集》。
        ~\\
        \songti

        \poem{254 西江月·雷氏辩论}{Mono6}
        \begin{center}
        \kaishu
        不合格律,但图一乐。
        
        \songti
        ~\\
        不懂装懂可悲,人云亦云可叹。
        
        问题错误且太差,把我理论细看。
        
        ~\\
        不愿证明无知,早就已经解释。
        
        慢慢理解要智商,再见告辞有事。
        \end{center}

        \newpage

        \poem{255 桂殿秋·仿雷绍武复读}{雷绍夫}
        \begin{center}
        下线了!再见个!
        
        好好阅读理解过。
        
        人渣辱骂太差的,
        
        不想和你们多说。
        \end{center}

        \poem{256 雷嘲}{普露普露}
        \begin{center}
        运交华盖欲何求,未敢诺奖棍碰头。
        
        破帽遮颜百禄过,漏船载酒岷中流。
        
        横眉冷对猪狗指,俯首甘为真理牛。
        
        躲进蕉林成一统,管他冬夏与春秋。
        \end{center}

        \section{257 反雷对联(其三十四)}
        \shanglian{Thunder Monkey}{一点误差,岂能忽略不计}
        \xialian{Thunder Monkey}{十年谬理,唯可证明无知}

        \poem{258 沁园春·反雷}{赵明毅}
        \lid

        时代更新,人类进步,万物安详。
        
        恨顽愚鼠辈,宣传谬论,民科老狗,无赖猖狂。
        
        井底之蛙,固执腐朽,坐镇猴山称帝皇。
        
        面观众,愈丢人现眼,实乃荒唐。
        ~\\

        余生只剩迷茫;俱老矣,黄昏路不长。
        
        叹立强金甲,古稀不化,文涛方士,病入膏肓。
        
        绍武雷神,十年论战,化作波涛入海洋。
        
        悲此等,浪费多少岁:一世彷徨。
        ~\\
        \lidend

        \poem{259 黄田坝怀古}{Sxlzr444}

        \kaishu
        昨日离开观测站,来到黄田坝考察,见红薯小麦之繁盛,忆雷公井盖之故事,有感而发。
        
        \songti
        \begin{center}
        红薯小麦黄田坝,遥想当年廿六八。

        井盖不与雷公便,方寸一盒沉泥沙?
        \end{center}

        \newpage

        \poem{260 咏雷·雷氏化学(其二)}{氢氦锂铍硼}
        \begin{center}
        凡氢电子皆等效,甲基电离是醋酸。

        NS组合大派键,两种结构环丙烷。
        
        相对运动原子静,电子云要跟着转。
        
        取代产物有六种,分子固定不能翻。

        ~\\
        \end{center}

        \poem{261 卜算子·咏雷}{芳華"}

        钢铁井盖下,绍武慌无主。七十七载空度日,却道有所成。

        有意研谬论,一任群士辱。纵使钻进下水道,难保清白故。
        ~\\

        \poem{262 卜算子·咏雷}{芳華"(街角小改)}

        钢铁井盖下,绍武慌无主。七十七载空度日,却道有所著。

        有意研谬论,一任群士辱。纵使钻进下水道,难保清白故。
        ~\\

        \newpage
        
        \poem{263 打雷一首}{Sxlzr444}
        \begin{center}
        黄田坝前逞武斗,窨井盖下缩其头。

        自少无知八二六,到老无德雷老猴。
        \end{center}

        \poem{264 折红英·反雷}{赵明毅}

        \lidl
        身荒废,何知退?固执迂腐如酣睡。
        
        近西归,益倾颓,陷泥潭下,终生无为:
        
        雷、雷、雷。
        ~\\

        高阳照,凭空眺,至肖公嘴欢歌绕。
        
        惜年耄,形枯槁,溺于谬论,不谙其好:
        
        绍、绍、绍。
        ~\\

        思维腐,入歧路,渐行偏远何时复?
        
        诚倾诉:尚能补!抛弃民科,晚年安度:
        
        武、武、武。
        \lidend

        \newpage

        \chapter{打油诗一组}
        \begin{center}
            (《反雷集》序号265\textasciitilde281)
        \end{center}

        \kaishu
        以下打油诗皆仿写自这首经典的反雷作品,作者:不拎皮箱的鱼。

        \songti
        \begin{center}
            雷老狗,雷老狗,咬住骨头不松口。
            
            前朝一为八二六,世代无耻雷老猴!
        \end{center}

        \poem{265 打油诗一组(其一)}{吉大化院老宋}
        \begin{center}
            雷老狗,雷老狗,不懂装懂常出丑。
        
            篡改数据挥挥手,辩得无言就逃走。
        \end{center}

        \poem{266 打油诗一组(其二)}{吉大化院老宋}
        \begin{center}
            雷老师,雷老师,自我催眠真无知。

            次次打脸不知耻,世人期盼进盒时。
        \end{center}

        \poem{267 打油诗一组(其三)}{雷绍夫}
        \begin{center}
            雷老狗,雷老狗,攥着井盖不松手。

            理论错误就改口,上窜下跳像只猴。
        \end{center}

        \poem{268 打油诗一组(其四)}{Mono6}
        \begin{center}
            雷老猴,雷老猴,分不清楚柱和轴。
            
            专家来了我逃走,质疑面前我缩头。
        \end{center}

        \poem{269 打油诗一组(其五)}{Thunder Monkey}
        \begin{center}
            雷老师,雷老师,产出真理手中持。

            人见真理齐说好,纷纷忙往田间施。
        \end{center}

        \poem{270 打油诗一组(其六)}{雷绍夫}
        \begin{center}
            雷老师,雷老师,自诩实践出真知。

            乐山桥头丢碎石,轻重齐落猴气死。
        \end{center}

        \poem{271 打油诗一组(其七)}{Thunder Monkey}
        \begin{center}
            \textbf{雷}院士,雷院士,\textbf{绍}真寻理有大智。
            
            \textbf{五}谷腹中化真理,\textbf{春}日入田润尘世。
        \end{center}

        \poem{272 打油诗一组(其八)}{Mono6}
        \begin{center}
            雷老鬼,雷老鬼,自我高潮不知北。
        
            千篇咏雷引以傲,不知其中多少伪。
        \end{center}

        \poem{273 打油诗一组(其九)}{Mono6}
        \begin{center}
            雷老汉,雷老汉,虚度一生令人叹。
            
            无知无耻一蚍蜉,竟言要把大树撼。
        \end{center}

        \poem{274 打油诗一组(其十)}{Mono6}
        \begin{center}
            八二六,八二六,早年作恶老遗臭。

            四无本性群中显,耍赖无人出其右。
        \end{center}

        \poem{275 打油诗一组(其十一)}{乐山的雷公}
        \begin{center}
            黄田坝,黄田坝,黄田坝中藏井下。

            井下深幽出真理,真理在手何所怕。
        \end{center}

        \poem{276 打油诗一组(其十二)}{Mono6}
        \begin{center}
            \textbf{雷}院士,雷院士,\textbf{真}理在手无所惧。
            
            \textbf{四}力一统大道简,\textbf{无}限光辉人才育。
        \end{center}

        \poem{277 打油诗一组(其十三)}{Thunder Monkey}
        \begin{center}
            黄田坝,黄田坝,坝下有井为龙床。
        
            床动龙静产真理,理入人心统炎黄。
        \end{center}

        \poem{278 打油诗一组(其十四)}{雷吟}
        \begin{center}
            雷老师,雷老师,斗酒百篇咏雷诗。

            篇篇暗藏耍猴字,民科必亡雷傻智。
        \end{center}

        \poem{279 打油诗一组(其十五)}{Mono6}
        \begin{center}
            运动力,运动力,一式欲把官科替。
        
            自诩雷论为真理,旁人视之如猴戏。
        \end{center}

        \poem{280 打油诗一组(其十六)}{我要删号}
        \begin{center}
            NS极,NS极,电子组合靠磁力。
            
            丙烷成环同极吸,阿Q精神属第一。
        \end{center}

        \poem{281 打油诗一组(其十七)}{cgnj512}
        \begin{center}
            雷老狗,雷老狗,百伏火线不电手。
            
            泼赖引得众人辩,无言以对忙遁走。
        \end{center}

        \chapter{}

        \section{282 反雷对联(其三十五)}
        \shanglian{雷氏力学课代表,ID君小改}{井(梦)里乾坤大}
        \xialian{雷氏力学课代表}{盒中日月长}

        \kaishu
        ID君:上联改为“梦里乾坤大”。一梦全能,蕴含了雷氏理论的一切,所以说乾坤大。而井里只有826和132的东西,明显不如梦里包含的东西多;盒中日月长是说雷老师动不动就进盒,似乎在那个盒子里中只有短短数小时,但却好似历经千年,而雷氏理论也历久弥新。
        \songti

        \poem{283 反雷}{3w.wmo}
        \begin{center}
            一生无所为,梦中信徒陪。
            
            全是荒唐言,能变白为黑。
        \end{center}

        \poem{284 反雷}{Mono6}
        \begin{center}
            夜晚居匣中,白天匿井底。
            
            井底三江过,水过携真理。
        \end{center}
        
        \poemwithst{285 回文诗}{小改雷绍武的《咏雷·三八》}{Mono6}
        \begin{center}
            雷公绍武人已疯,绍武已疯茅坑中。
            
            已疯坑中还乱吠,坑中乱吠是雷公。

            ~\\
        \end{center}
        
        \noindent \textbf{附原诗:}
        \begin{center}
        \heiti \Large 咏雷\ \ 三八

        \kaishu《回文诗》回复H云淡风清Z
        \songti \large
        \end{center}
        \begin{center}
            雷绍武
        \end{center}

        \begin{center}
        云淡风清人已疯,清人已疯茅坑中。

        疯茅坑中乱狂吠,中乱狂吠云淡风。
        \end{center}

        \newpage

        \poemwithst{286 清平乐}{小改雷绍武的《咏雷·四二》}{Mono6}

        雷神已老,晚节更难保。无数谬言不可考,自造笑谈不少。

        作咏雷自高潮,遇反对则叫嚣。小丑班门弄斧,一世井下哀嚎。
        ~\\

        \noindent \textbf{附原诗:}
        \begin{center}
        \heiti \Large 咏雷\ \ 四二

        \kaishu 清平乐(回云淡风清)
        \songti \large
        \end{center}
        \begin{center}
            雷绍武
        \end{center}

        雷神虽老,晚节更辉煌,自创理论实不少,处处证据可考。

        忽遇弱智围攻,冷静横扫从容,胸有无敌真理,岂怕苍蝇嗡嗡。

        \poem{287 七律·题雷吧}{鸣雷海战}
        \begin{center}
            先师新著满店香,挺雷压笑唤客尝。

            酒醉犹嫌神论晚,涌雷千首各尽觞。

            教授杂猴齐骂战,走狗科奴笑满堂。

            应念猴山能济世,岂能人走便茶凉!
        \end{center}

        \poemwithst{288 十六字令·反雷}{第一届“井盖杯”模拟赛优秀作品(13分)}{吉大化院老宋}

        \begin{center}
            雷,胡扯多年梦飘飞。何其傲,潭里陷难归。
            
            雷,病入膏肓不可回。行猴戏,观众后相随。
            
            雷,年近八十未作为。犹难醒,顽固令人悲。

            ~\\
        \end{center}

        \poemwithst{289 十六字令·反雷}{第一届“井盖杯”模拟赛优秀作品(12分)}{群助手}

        \begin{center}
            雷,臆测科学大笔挥。颅内想,颠倒是和非。

            雷,自我催眠壮胆威。一掌醒,小丑面容回。
            
            雷,论战四天大势颓。颜面扫,装死告辞归。
        \end{center}

        \newpage

        \poemwithst{290 Antilei 18 (English Translation)}{第一届“井盖杯”模拟赛优秀作品(12.5分)}{拉普雷斯 Laplece}

        Lei\footnote{读轻声,近似于读``The"。} monkey is out of case, (绍武刚出笼)

        and wanna play in front of us. (又欲嬉于众)

        Delicious though a banana is, (香蕉虽珍馐)

        virtue should be kept in his minds\footnote{考虑到说不准原理,可认为雷老师有多个mind提出理论,故用复数minds。}! (应以晚节重)

        \poem{291 黄田坝之战}{Sxlzr444}
        \begin{center}
        清水河畔,黄田坝前,阳光明媚。
        
        一三二厂,大国重器,铁鸟高飞。
        
        遥想当年,激情岁月,雷公逞威。
        
        入八二六,手持真理,英勇无畏。
        
        战产业军,当枪与炮,以棍相对。
        
        力战滩头,血染河畔,不曾言退。
        
        运动力小,寡不敌众,势颓况危。
        
        见窨井盖,急中生智,入以自卫。
        
        魔高一尺,道高一丈,平安而回。
        
        赞真理雷,明哲保身,急流勇退。
        \end{center}

        \poem{292 何其歌(其一)}{赵明毅}
        \begin{center}
            绍武何其差,说不过就骂。
            
            众人敌不过,躲到井盖下。

            ~\\
        \end{center}

        \poem{293 何其歌(其二)}{乐山的雷公}
        \begin{center}
            绍武何其蠢,妄把官科损。
            
            无知又无德,撒泼又打滚。

            ~\\
        \end{center}

        \poem{294 何其歌(其三)}{Mono6}
        \begin{center}
            雷公何其智,十年如一日。
            
            潜心游学海,真理终出世。
            
            真理何其明,春秋大梦成。
            
            颅内观天下,一梦可全能。
        \end{center}

        \newpage

        \poem{295 何其歌(其四)}{雷绍夫}
        \begin{center}
            抛友弃理想,独身逃井下。
            
            雷猴何其差,我都不想骂!

        \end{center}

        \poem{296 何其歌(其五)}{Mono6、ID君}
        \begin{center}
            绍武何其坏,逢人耍泼赖。
            
            辩论十连败,无耻钻井盖。
            
            井盖何其宽,绍武往里钻。
            
            钻完就再见,再见等明天。
            
            明天何其多,日日有学说。
            
            前后相矛盾,我也不背锅。
        \end{center}

        \poem{297 何其歌(其六)}{Sxlzr444}
        \begin{center}
            井盖何其好,雷猴小命保。
            
            当年八二六,棍棒对枪炮。
            
            料定不可当,钻井把身藏。
            
            如今性不改,仓皇进盒跑。
        \end{center}

        \poem{298 反雷·九眼桥}{Sxlzr444}
        \begin{center}
            九眼桥下神通大,军官学士辩真假。
            
            四年武斗成蹉跎,不如此地把钱花。
        \end{center}

        \poemwithst{299 井研是个好地方}{改编自雷绍武作词的《乐山是个好地方》}{Mono6}
        \begin{center}
        井研是个好地方
        
        井下研理雷家乡
        
        雷家乡
        
        山美水美猴更美
        
        文明观猴乐陶陶
        
        乐陶陶
        
        ~\\
        三江真理绕井流
        
        绍武缩在井中央
        
        绍武缩在井中央
        
        井底真理美如画
        
        躲避棍棒好地方
        
        ~\\
        文学科学双遗产
        
        贴吧旅游目的地
        
        最好地方
        
        千篇咏雷保平安
        
        万载公转送吉祥
        
        送吉祥
        
        ~\\
        粮食猪肉菜籽油
        
        骆驼奶粉溢芳香
        
        骆驼奶粉溢芳香
        
        八二六武斗趣味多
        
        春秋五度故事长
        
        ~\\
        井研是个好地方
        
        电子梦里都向往
        
        都向往
        
        运动新论通五洲
        
        雷氏电子联成网
        
        联成网
        
        ~\\
        除蛆保雷促发展
        
        一年更比一年好
        
        一年更比一年好
        
        井下赞歌唱不完哟
        
        我爱井研爱雷乡
        
        ~\\
        我爱井研爱雷乡
        
        我爱井研爱雷乡

        ~\\
        \end{center}

        \poem{300 蝶恋花·反雷}{赵明毅}

        厂外呼声传万里,数百学生,人海惊涛起。枪响应声旁者毙,惊逃井下黄土里。

        半百年来存记忆,人似从前,鼠胆无一技:论战数天常败绩,层出借口将门闭\footnote{原作者反复考虑是“将门闭”好还是“将盒闭”好。古有韩昌黎思考“推”“敲”,今有赵大锑斟酌“盒”“门”。作者提议,在反雷创作中,把推敲遣词用字的行为称作“盒门”}。



\end{document}