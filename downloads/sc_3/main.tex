
%%%%%下面这些都是全文的配置,别乱动就是了
\documentclass[UTF8,12pt,oneside]{ctexbook}
\usepackage{graphicx}
\usepackage{xeCJK}
\usepackage{indentfirst}
\usepackage{titletoc}
\usepackage{fancyhdr}
\usepackage{fontspec}
\usepackage{geometry}
%\geometry{left=2.5cm,right=2.5cm,top=2.5cm,bottom=3.0cm}
\setmainfont{Times New Roman}
\pagestyle{plain} % 此处为fancy时有页眉
%%%%%上面这些都是全文的配置,别乱动就是了

%%%%%这是封面页
\title{\textbf{\fontsize{42}{84}{反雷杂志 \\[15] Anti-Lei Magazine}}}
\author{\Large \kaishu 雷氏力学吧、太差太差吧内部资料 \\[8] \Large \kaishu 《反雷杂志》编辑部 }
\date{\huge 2021年6月第1期 \\ 反雷古代文学专刊 \\ (总第3期)}

%%%%%这里定义了宏
\def\pau#1{\begin{center} {#1} \end{center}} %第一个宏:诗的作者
\def\pst#1{\begin{center}\Large \kaishu {#1} \songti \large \end{center}}%第二个宏:诗的副标题
\def\poem#1#2{\section{#1}\pau{#2}} %第三个宏:标题+作者
\def\poemwithst#1#2#3{\section{#1}\pst{#2}\pau{#3}} %第四个宏:标题+副标题+作者
\def\shanglian#1#2
    {\noindent{\textbf{上联:}({#1})}

    \begin{center}
        {#2}
    \end{center}
    }

\def\xialian#1#2
    {\noindent{\textbf{下联:}({#1})}

    \begin{center}
        {#2}
    \end{center}
    }
\def\lid{\setlength\parindent{5em}}
\def\lidlid{\setlength\parindent{8em}}
\def\lidend{\setlength\parindent{2em}}

\begin{document}
    
    \maketitle %生成封面页
    
    %%%%% 目录配置,请勿改动
    \setcounter{secnumdepth}{-2} 
    \setcounter{tocdepth}{1}
    \titlecontents{section}[16pt]{\addvspace{2pt}\filright}
    {\contentspush{\thecontentslabel\hspace{0.8em}}}
    {}{\titlerule*[8pt]{.}\contentspage}
    
    \chapter{刊首语}
    
    \Large

    \begin{center}

        ~\\

        \noindent 看绍武,\textbf{顽固耻辱恼羞卑微放肆狼狈迷惑疯癫昏聩},
        
        何其\textbf{孤独},使人\textbf{痛惜悲哀},\textbf{遗落}于人世。
        
        ~\\
        \noindent 笑雷猴,\textbf{愚蠢软弱腐朽邪恶猖狂狡诈渺小蛮横贪婪},
        
        不知\textbf{羞耻},理当\textbf{灭除诛杀},\textbf{斩决}当何时?

        ~\\
    
    \end{center}

    \large
    \kaishu
    \hfill{——(134)《山坡羊·反雷其二十七》题记,赵明毅}
    \songti

    \chapter{本期聚焦}
        \begin{center}
            \Large
            \kaishu
            ~\\
            赵明毅《山坡羊·反雷》系列二十七首 (第23页)
            \\ 另有《扇泼猴》若干首
            
            ~\\
            反雷对联二十四副 (第37页)
            
            \songti
            \normalsize
        \end{center}
    
    \tableofcontents %生成目录
    
    \chapter{反雷集(76\textasciitilde107)} %新章节:反雷诗词
    
    \large %本部分使用大号字
    
    \poemwithst{76 沁园春}{小改咏雷版}{RobL61} %有副标题,使用\poemwithst
        
        \lid
        
        民科邪教,彼时嚣张,今且难逃。
        
        看神州上下,民智大开。绍武小丑,但供耻笑。
        
        脑中邪说,口中谬论,心怀粪土却自豪。
        
        更有甚,实偏执疯狂,自比天高。 
        ~\\
        
        理论毫无成就,只能臆管科正飘摇。
        
        看早时无知,尚有礼节。近年辱骂,自以为傲。
        
        世间科学,万千理论,唯有雷理无人瞧。
        
        心狭隘,看千篇咏雷,自我高潮。
        
        \lidend

        
    \newpage %根据需要,直接开始新的一页
        
    \poem{77 咒雷}{勒卉更} %没有副标题,使用\poem
                       \begin{center}
                       雷宅风景坏,日日伪科海。

                       老骨不识抬,终终唱戏来。 

                       后有反声起,智商如蝼蚁。 

                       早晚抱佞理,进入盒子里。
                       \end{center}
        
        %注:用\footnote插入脚注
     %强制换行符,用这个符号可以实现多个空行
        
    \poem{78 打油诗·抓猴}{勒卉更} %没有副标题,使用\poem
        
        \begin{center} %居中排版
        四川有乐山,山上有只猴。

        猴子太顽皮,天天来叫喊。

        大家来努力,把猴抓一起。

        抓来放哪里?放到盒子里。

        \end{center}
    
    \poem{79 定风波·三劝雷绍武}{RobL61}

    绵绵细雨春意浓,凄美时节酒一盅,再醉不及老猴勇。愚勇,蚍蜉撼树无人崇。 

    风卷残云落叶休,莫急,跳梁小丑自会怂。民科邪教无人从,劝雷,早日从真不卖弄。 
    
    \newpage
    
    \poem{80 水调歌头·反雷}{赵明毅}
    
     \lid
       万数谬论创,四海恶名扬。
       
       无端捏造,论述皆是臭而长。
       
       两种人才不懂,四有者全支持,梦里造天纲。
       
       坚持十余载,不化是雷郎! 
       ~\\

       肉电子,高温下,转成光。
       
       颅中实验,臆测歪理愈猖狂!
       
       劝汝回头是岸,不致晚节不保,专注写文章。
       
       顽固何须问?理解要智商!
       
    \lidend
      
      ~\\
      
    \poem{81 诉衷情·反雷}{RobL61}
    \lidlid
    
        当年自贱成老猴,求人把蕉投。
        
        人智断于何处?井盖下缩头。 
        ~\\
        
        盒未进,智先流,鬓已秋。
        
        人生如梦,猴戏十载,梦断嘉州。
    
    \lidend
        
    \newpage

    \poem{82 反雷}{雷绍夫}
     \setlength\parindent{6em}
     
         乐山之君子,井研之\textbf{雷}公。
         
         八二六时\textbf{候},洁身而自好,不与恶徒共。
         
         俯身三十载,屈尊下井盖,\textbf{为}保真理在。 
         
         后又出山,据理斥管科,有如\textbf{老}猴跃盒中。
         
         \textbf{不}畏无德人,教化电子众。
         
         \textbf{尊}为雷老狗,力学传千秋。
         
         知识\textbf{无}边界,诸学皆精通。
         
         管科\textbf{知}其论,无人不震悚 。
         
         牛爱若能闻,\textbf{无}地以自容。
         
         \textbf{德}高而望重,雷电于长空。

     \lidend

\poem{83 一剪梅·五劝雷绍武}{RobL61}
\begin{center}
    科妄少智太无脑,颅涌高潮,心狂自傲。 

    谬理空谈创邪教,自比天高,出丑见笑。 

    劝君心沉莫急躁,勤学思考,皆不可少。 
    
    夜郎自大勿再效,少盼香蕉,多学正道。


    \end{center}

\newpage
\poem{84 江城子·四劝雷绍武}{RobL61}
    
    疯者狂人事无成,以史鉴,皆可证。遥想独逸\footnote{独逸:即德国。},“大梦”被人憎。草菅人命败战争,山河碎,后世讽。 
    
    复劝绍武莫要疯,谬理论,无人奉。自比天高,科妄空想登。谨记恩师教诲赠,勤学习,白头等。
    ~\\

\poem{85 永遇乐·反雷}{乐山的雷公}

    千古江山,歪理无觅,雷绍武处。经典力学,全盘否定,妄言运动力。百万误差,忽略不计,地球转动绕柱。看电子,正负南北,串联并联成组。
    
    钻研十载,歪理邪说,独行民科长路。思而不学,劝而不听,年老愈发顽固。回首当年,不学无术,年少入八二六。凭谁问,绍武老矣,能诺奖否。

\newpage

\poem{86 渔家傲·反雷}{赵明毅}

    万千理论皆狗屁,老雷无耻弄猴戏,十载坚持从未腻。脑子里,不知屎尿何容器?
    
    自恃清高视天地,迷之自信忽非议,可笑至极仍蒙蔽。黄土里,棺材一口为栖地。
    ~\\

\poem{87 采桑子·反雷}{赵明毅}

    嘉州处处风景好,独有一糟。独有一糟:雷狗邪说叫喊高。 
    
    传播谬论批真理,放弃贞操。放弃贞操,猴戏民科终遁逃。
    ~\\

\poem{88 浣溪沙·反雷}{赵明毅}

    学界和平数百年,一只小丑跳出天。摆出歪理舞翩翩。 
    
    无耻宣扬歪谬论,自大忽视好心言。重归正轨亦何难?
    ~\\

\newpage

\poem{89 临江仙·反雷}{赵明毅}

    不知信众皆观戏,以为自悟天机。可怜年至古来稀,智商低下未能及。
    
    身况渐微何可怜,内心昏聩低迷。人寰撒手首神离,只添玩笑供人提。

\poem{90 讨咏雷派檄文}{雷专打魔怔人}
\begin{center}
    芸芸众生,失理则泠;蒸蒸灵息,无仪则獍。
    
    人有温热,天有暗明;荣于善行,耻于恶名。

    乐山雷氏,不仁劣性;世居岷江,无德无情!

    当思官科至理,一统真理盛兴。

    拭目今日,民科横行;

    道不能畅,理不得清;

    上无所申,下入窘境。

    不守节操,造作成性。

    食人之不敢食,饮人所未尝饮,

    弃德之昭明,抛心之魂灵,

    阳为人之形状,阴则阿谀后庭。

    登堂妒忌,吧友争相沽名;

    入室纷争,以讨雷氏高兴;

    谗语掩杀,真理怏怏无命。

    怂恿污流,演义肮脏之风;

    邪说万能,可以不要魂灵。

    无知,卖笑纵情;

    无德,抛却亲情;

    无耻,丧尽人性;

    无赖,远离友情。

    如今神人共怒,天地更所不容。

    犹有暗藏祸心,妄想天崩。

    以达乱中取胜,饱囊别宫。

    科学之蝥虫,祸害已成;

    乐山之贼子,何其求成?

    噫嘻!反雷幡然悟醒,

    独尊官科至圣。

    柳丝飘拂,觉春来之绿生;

    荷珠映彩,喜夏花之幽净。

    愚一介布衣,无所适名。

    谨奉宗旨,循真理之志,演反雷之风。

    气冲云天,立志长缨!

    顺乎人心,爰举义旗!

    吁呼之诚,竭尽所能!

    但求清风,呼唤真理,呐喊人性!

    复苏复苏兮,真理!回归回归兮,人性!

\newpage

\end{center}
\poem{91 雷猴理论来源}{析境}
\begin{center}
    篡改事实,混淆是非;颠倒黑白,不辨真假。
    
    无理无据,无知无赖;完全错误,太差太差。

\end{center}

\poem{92 猴道难}{雷公助我}

    噫吁嘻,雷猴叫哉!猴道之难,难于进盒雷\footnote{进盒雷:指雷老师探寻真理,进入真理的宝盒。}!运动\footnote{运动:这里指雷老师的运动力理论。}与电子,雷力何茫然!尔来七十又几岁,不与管科通人烟\footnote{通人烟:指说人话。}。民科绍武有猴道,可以四无又疯癫。
    
有知有德壮士生,然后正确中出\footnote{这里“中出”存在歧义,有人说“支持”也可。}相钩连。上有运动速度\footnote{运动速度:运动的一定是物体的速度,这里是符合理论的。}之高标,下有SNNS之回川。管科之技尚不得过,猿猱下床床攀援\footnote{猿猱下床床攀援:猴子从床上攀援而下,而实际是床从猴子上攀援而上。}。川渝何盘盘,川大之才不中干。缩头进井仰胁息,恩师正贤坐长叹。

 问君地球何时还?五〇二七九〇三。只见雷进八二六,捂头乱窜绕井间。又闻扣饭心不甘,书记难。猴道之难,难于上青天,使人听之没脸见。
 
全吧反串新天地,雷公一梦做上天,发帖对线争喧豗,看不清楚实验雷。其脑也如此,嗟尔逗猴之人胡为乎来哉?

 运动力学绍武开,一夫当关,万夫备盖\footnote{盖:代指上文“盒”,即真理的宝盒。}。所守之理论,化为火与柴。朝避吧友,夕编咏雷,磨牙吮血,杀人如麻。贴吧虽云乐,不如早还家。猴道之难,难于真理添!楼主西望长咨嗟!

\poem{93 青玉案·反雷}{赵明毅}

    七十人语\footnote{语:yù}从心顾,仅雷老,行歪路。人世百年何短苦?怜君未有,利他之处,唯显无为碌。
    
    星辰日月流如顾,但耻一生陷深误!日堕黄昏蒙万物。若求良谥,行将就木,及早歧途住!

\poem{94 蝶恋花·六劝雷绍武}{RobL61}

    武斗大败井盖跑,举止荒唐,狼狈太可笑。四年虚度学习少,净行不义何处逃? 
    
    年高体衰心未老,十年表演,何苦设囚牢?劝君莫要恋香蕉,当以人脑换猴脑。

\newpage
\poem{95 江城子·反雷 }{赵明毅}

绍武何曾解科学,智商缺,海内绝。爬至零线,猴屁股长撅。现象证明雷论错,不听劝,意坚决。 
    
杨公应恨教此学,智常竭,把d约。只盼回头,能把谬真抉,如是经年入土后,可含笑,保操节。

\poem{96 卜算子·反雷}{赵明毅}

少时战斗团,老矣传歪理。历史汪洋如微石,未作涟漪起。 
    
如梦前半生,似幻终年里。腐朽残身含笑终,感叹无悲喜。

\poem{97 水调歌头·一评雷绍武 }{RobL61}

    狂人在何处?乐山井研有。笃信颅内大梦,自贱成老猴。诚心相劝不顾,冷嘲热讽取辱,疯狂令人呕。狂犬尚吠日,处世如老狗。 
    
智渐衰,言益愚,让人忧。不学无术,怎见恩师于身后?少年作恶武斗,晚节自毁心魔,何苦要自囚?枉度七十载,哀乐盒中奏。

\newpage
\poem{98 风入松·二评雷绍武}{RobL61}

    无知老朽常扮猴,晨起拭眼眸。电子火花到处透,零线无流,无可药救。但笑牛顿无知,意淫官科无谋。
    
地球公转万年久,反对皆荒谬。沽名钓誉成小丑,狂吠似狗,乐上心头。早年荒废学业,晚年频频上钩。

\poem{99 怒雷}{勒卉更 }

    千古悠悠,顽莫如,雷绍武。蔑视真理唱劣戏,假大师,真无知! 
    
傻论频出仍不耻,乐山桥上实验试。无知无耻至如此。可笑!何胆还作咏雷诗!

\poem{100 钗头凤·七劝雷绍武}{RobL61}

    人情险,世情恶,谬理无人认真谈。不听劝,但心寒。愈发偏执,颅内空研。叹,叹,叹!
    
心正端,少臆想,收拾朽脑向前看。多纳言,莫等闲。迷途知返,师欣九泉。返,返,返!

\newpage
\poemwithst{101 南乡子·八劝雷绍武 }{改编自震惊百里的南乡子·咏雷 }{RobL61}

    名利何处求? 但来雷吧仙府游。 一发谬论百人呼。 吹牛, 沽名钓誉实可忧。 
    
弃名少上钩, 安度晚年过伞寿。 何苦自贱成老猴? 心愁, 劝君退隐褪污垢。

~\\
\poem{102 钗头凤·n讽雷绍武}{小野寺麗}

    伏生碌,囊萤馥,义生贤士功德铸。雷人悟,愚言入,几多讹误,谬答如注。怒,怒,怒。
    
年迟暮,邪心固,一生时日空虚度。黄泉路,风潇簌,视其尤库,竟将毒\footnote{毒:指826}助。覆,覆,覆。

\newpage
\poem{103 扇泼猴}{乐山的雷公}       

    岷江东去,乐山四顾,世间谬理在何处。看雷公,谈理数,胸无点墨仍顽固,歪理邪说如粪土。蠢,雷绍武,猴,雷绍武。

\poem{104 扇泼猴·山坡羊咏雷其二十二改}{乐山的雷公}

    懒惰如猪,无力似鼠,乐山老朽性迂腐。似八婆,似怨妇,安敢把那贤能妒,真理面前似若侏儒!愚,雷绍武,蠢,雷绍武!

\poemwithst{105 扇泼猴}{改自山坡羊·一劝赵明毅}{乐山的雷公}

    楚歌四面,孤城紧闭,民科小丑势已去。谬误出,歪理提,负隅顽抗不量力,一败涂地已成定局!劝,早日拥真理,诫,早日拥真理!

\newpage
\poem{106 扇泼猴}{Thunder Monkey}

\setlength\parindent{11em}

    \textbf{四}年谬误,
    
    \textbf{川}中硕鼠,
    
    \textbf{井}下曾为栖身处。
    
    \textbf{研}万物,只在颅,
    
    \textbf{雷}理不传心中苦,
    
    \textbf{犬}牙乱咬恰似泼妇。
    
    \textbf{狂},雷绍武!
    
    \textbf{吠},雷绍武!

\poem{107 扇泼猴}{Thunder Monkey}

    \textbf{乐}土天府, 
    
    \textbf{山}水如故, 

    \textbf{雷}动风涌官学固。 

    \textbf{犬}声出,心中苦, 

    \textbf{终}其一生难明悟。 

    \textbf{成}事不足谁与为伍,

    \textbf{奸},雷绍武; 

    \textbf{佞},雷绍武!

\lidend

\chapter{赵明毅《山坡羊·反雷》系列}

\begin{center}
    (《反雷集》序号108\textasciitilde134)
\end{center}

\subsection{山坡羊·反雷系列总序:}
  
\kaishu
面对无耻无赖的涌雷电子“乐山的雷公”的山坡羊·咏雷,我,赵明毅在此与之斗曲,全写山坡羊,至今更败一筹,特将这些篇目发表出来。同时,如果雷绍武本人看到了的话,希望你多看看《山坡羊·反雷其十三》,这是我的真心话,希望你迷途知返,这样以后你不多的岁月结束后也能有个好点的名声。
    ~\\

\songti

\poem{108 山坡羊·反雷其一}{赵明毅}

    盒中有物,原为绍武,张牙舞爪面容怖。智如猪,性轻浮,三千歪理皆发布,曾有哪篇无谬误处?愚,雷绍武;蠢,雷绍武!

\newpage

\poem{109 山坡羊·反雷其二}{赵明毅}

    雷猴拥簇,迷人注目,邪说每读欲呕吐。惜智无,望前途,伤心年至古稀处,活在梦幻自作论主。腐,雷绍武;朽,雷绍武!
~\\

\poem{110 山坡羊·反雷其三}{赵明毅}

    贫屋独处,盒中漫步,反雷电子无穷数!风云起,天地怒,民科无智行尽路,邪说背道终为灰土。邪,雷绍武;恶,雷绍武!
~\\

\poem{111 山坡羊·反雷其四}{赵明毅}

    阴魂无属,坚持如故,雷猴坏水填满腹。罪当诛,灭恶徒,猴戏再耍年四五,可笑民科即将作古。猖,雷绍武;狂,雷绍武!

\newpage

\poem{112 山坡羊·反雷其五}{赵明毅}

    不懂算数,头脑似木,学识短浅令捧腹。骂他徒\footnote{他徒:其他人},是科奴,若在其脑细细顾,闹虫跑狗飞鸟行鼠。狡,雷绍武;诈,雷绍武!
~\\

\poem{113 山坡羊·反雷其六}{赵明毅}

    真理天斧,劈裂邪物,决斩雷猴如毙兔。民科徒,势已孤,首领已行将就木,一朝树倒猢狲跑路。灭,雷绍武;除,雷绍武!
~\\

\poem{114 山坡羊·反雷其七}{赵明毅}

    神统万古,科学终普,造福苍生幸福路。民科猪,绍武出,反对科学骂学术,井底之蛙蚍蜉撼树。渺,雷绍武;小,雷绍武!

\newpage

\poem{115 山坡羊·反雷其八}{赵明毅}

    不知何故,打翻陈醋,面对真学甚嫉妒。欲荼毒,著邪书,谬论皆入绍武肚,妄图撼动科学做主。蛮,雷绍武;横,雷绍武!
~\\

\poem{116 山坡羊·反雷其九}{赵明毅}
    
    德行如鼠,猖狂如虎,天下皆望尸肉腐!凭栏哭,意踌躇:民科本属无耻物,何时进盒化为白骨?!贪,雷绍武;婪,雷绍武!
~\\

\poem{117 山坡羊·反雷其十}{赵明毅}

    不信科普,傻论无数,百劝执迷仍不悟。论若输,便言辱,无理取闹众人怒,手捧谬论仍试修补。顽,雷绍武;固,雷绍武!

\newpage

\poem{118 山坡羊·反雷其十一}{赵明毅}

    不知万物,不识一数,学者面前弄大斧。自傲殊,自信笃,秒遭打脸如吃醋,酸苦急躁出言侮辱。恼,雷绍武;羞,雷绍武!
~\\

\poem{119 山坡羊·反雷其十二}{赵明毅}

    思维迂腐,无耻如鼠,小丑妄把科学阻。名声污,智不足,众叛亲离不醒悟,怀怨井蛙终将入土。孤,雷绍武;独,雷绍武!
~\\

\poem{120 山坡羊·反雷其十三}{ 赵明毅}

    行将就木,仍行歪路,渐行渐远未停步。明诗书,知今古,只因无知有此故,弃明投暗走入迷处。痛,雷绍武;惜,雷绍武!

\newpage

\poem{121 山坡羊·反雷其十四}{ 赵明毅}

    不堪细读,不认谬误,雷氏歪理马脚露。咏雷徒,势已孤,历史茫茫如尘土,一切民科定入棺木。卑,雷绍武;微,雷绍武!
~\\

\poem{122 山坡羊·反雷其十五}{ 赵明毅}

    地球转速,百万计数,滑稽理论使捧腹。零线出,被零除,前人理论以万数,皆被民科碰瓷围堵。放,雷绍武;肆,雷绍武!
~\\

\poem{123 山坡羊·反雷其十六}{赵明毅}

    真理如虎,天震人怒,科学卫军凌江渡!民科除,大道复!正论之枪所到处,邪说信徒奔命似鼠!狼,雷绍武;狈,雷绍武!

\newpage

\poem{124 山坡羊·反雷其十七}{赵明毅}

    自称朴素,实则朽腐,陈旧观念心底住。欲高呼:“回正途!”反被自大言侮辱,无奈只等猴自作古。迷,雷绍武;惑,雷绍武!
~\\

\poem{125 山坡羊·反雷其十八}{赵明毅}

    数典忘祖,出言相辱,何能将汝谬论补?似狗扑,狂叫呼,毫无知识作基础,雷猴撒泼直至入土。疯,雷绍武;癫,雷绍武!
~\\

\poem{126 山坡羊·反雷其十九}{赵明毅}

    劝言不入,坚持如故,装死无能发狂怒。写歪书,难通读,数物化文皆涉处,万科全无知识基础。昏,雷绍武;聩,雷绍武!

\newpage

\poem{127 山坡羊·反雷其二十}{赵明毅}

    生于巴蜀,养于天府,浪费栽培入歧路。问苍天,此何如?人类败类是此~徒,不肖子孙民科硕鼠!羞,雷绍武;耻,雷绍武!
~\\
    
\poem{128 山坡羊·反雷其二十一}{赵明毅}

    若有出入,物体错误,多次修改雷常数。视频出,不相符,便说时间秒秒顾,如墙头草随风簌簌。软,雷绍武;弱,雷绍武!
~\\
    
\poem{129 山坡羊·反雷其二十二}{赵明毅}

    自望天府,愁水满腹:忠言绍武耳不入。怜情抒,泪沾服:古稀人生因此输,死抓谬论不肯降服。悲,雷绍武;哀,雷绍武!

\newpage
    
\poem{130 山坡羊·反雷其二十三}{赵明毅}

    故人已去,空留雷鼠,众叛亲离众人辱。入迷途,友难阻,人生岔路一再误,何其不惜感人肺腑?遗,雷绍武;落,雷绍武!
~\\
    
\poem{131 山坡羊·反雷其二十四}{赵明毅}

    不惠绍武,尽惹人怒,且听大军铿锵步!真理出,科信徒,自成卫队捍正主,科学浪潮涤净巴蜀!斩,雷绍武;决,雷绍武!
~\\
    
\poem{132 山坡羊·反雷其二十五}{赵明毅}

    硕鼠硕鼠,无食我黍!万人共愤民科属。雷猴徒,欲反扑,怎敌真知科学住,灰飞烟灭化为尘土!诛,雷绍武;杀,雷绍武!

\newpage
    
\poem{133 山坡羊·反雷其二十六}{赵明毅}

    人生短苦,少有命主,百年能成大旗鼓。然雷徒,十年扑,竟成此多歪论著,无限笑料流传万古!耻,雷绍武;辱,雷绍武!
    
\poem{134 山坡羊·反雷其二十七}{赵明毅}
    \subsection{题记:}
    \normalsize
    \kaishu

    若涌雷电子按照承诺把他的《山坡羊·咏雷其二十六》作为封笔之作,这将会是山坡羊这一组的曲的最后一篇。

    前二十六首的最后两句“雷绍武”前一个字连在一起组成26个词,罗列在此,这将是反雷电子声势浩大地举起血红的科学之旗的序幕!

    \songti

    \begin{center}

    \noindent 看绍武,\textbf{顽固耻辱恼羞卑微放肆狼狈迷惑疯癫昏聩},
    
    何其\textbf{孤独},使人\textbf{痛惜悲哀},\textbf{遗落}于人世。

    \noindent 笑雷猴,\textbf{愚蠢软弱腐朽邪恶猖狂狡诈渺小蛮横贪婪},
    
    不知\textbf{羞耻},理当\textbf{灭除诛杀},\textbf{斩决}当何时?

    \end{center}

    \subsection{正文:}
    \large
    千军拥护,万人声促,绍武进盒当从速!雷学徒,尽皆诛,大道真理如震怒,科论万宗终将光复!奔,雷绍武;亡,雷绍武!
    
    \newpage

    \kaishu
    另附诗一首,以彰对山坡羊斗诗中,反雷电子收官之作的完结:(此诗编入反雷·170)

    \songti
    \begin{center}
        \normalsize
        黄田坝前白日昏,千重骇浪犹腾奔。

        绍武幼时争战地,往来种作和气存。

        浩劫当年事堪叹,绍武试把天地撼。

        老来重拾翻覆志,谬论层出妄开天。

        乐山冻饥民科惨,理论不进渐成殇。

        老矣偏执不听劝,愁容满面临岷江。

        争论落败所自致,贴吧奔窜如亡羊。

        堆床十斛仅麦屑,一勺入口无米汤。

        当时狂论意何取,离离满目悲禾黍。

        终有一日化白骨,黄土棺中朽体腐。

        若有人忆耍猴事,或见陆续来吊古。

        傻论笑料终消散,鲸波蚀尽战场土。
    \end{center}

    \large

\chapter{反雷集(135\textasciitilde143)}

\begin{center}
    (《扇泼猴》作品若干首)
\end{center}

\poem{135 扇泼猴·其一}{RobL61}

脑中空空,臆想汹汹,胡编谬理太匆匆。雷隆隆,思淙淙,沽名钓誉心彤彤,邪说尽出绍武口中。灭,民科从。斫,民科从。
~\\

\poem{136 扇泼猴·其二}{RobL61}

骄躁老猴,血沥心呕,但为谬理常奔走。躯已朽,几时休?真理大厦尚未否,小丑民科终成老狗。卑,民科猴;微,民科猴!

\newpage

\poem{137 扇泼猴·其三}{RobL61}

惛怓老猴,言语不周,猴言猴语使人逗。八二六,井盖投,年至古稀智商忧,下场凄惨自取咎由。灭,民科猴!绝,民科猴!

\poem{138 扇泼猴·其四}{RobL61}

见香蕉投,不顾衰朽,费尽心机太鄙陋。人劝汝,不思咎,反成狂犬咬人手,为人所殴方井盖投。贱,民科猴!悲,民科猴!

\poem{139 扇泼猴·其五}{RobL61}

疯癫绍武,言语恶毒,臆想管科已被除。人皆奴,唯尔悟,自诩学者知识无,民科小丑但供凌辱。骄,雷绍武;躁,雷绍武!

\poem{140 扇泼猴·其六}{RobL61}

绍武跋扈,癔症反复,傻脑当中谬理驻。劝不顾,把蕉护,自设囚笼盒里住,畜言狂叫恩师不孚。疯,雷绍武;癫,雷绍武!

\newpage

\poem{141 扇泼猴·其七}{RobL61}

自设桎梏,晚节不顾,遇见香蕉癫病复。望不孚,傻如故,一生光阴便虚度,可怜民科尚未悔悟。猴,雷绍武;戏,雷绍武!
~\\

\poem{142 扇泼猴}{Mono826}

无耻狂吠,无能狂怒,民科绍武如泼妇。德也无,知也无,今日放肆传谬误,明天就要往盒里住!悲,雷绍武;惨,雷绍武!
~\\

\poem{143 扇泼猴}{Mono826}

克服险阻,开辟通途,扫尽一切民科奴!邪说除,正理普,喷水飞机约d术,与你一并作了古!消,雷绍武;除,雷绍武!

\chapter{反雷对联}
\begin{center}
    (《反雷集》序号144\textasciitilde167)
\end{center}

\subsection{序:}

\kaishu
反雷对联的创作,是由Mono6(特别声明:与上述Mono826无关)发起,编辑部全员参与,并广泛吸收大群群友意见的一项创作活动。对联的创作相对于诗词可能较简单,但文学性未必亚于诗词,读来亦朗朗上口。热烈欢迎大家参与到反雷对联的创作中来。
\songti

\section{144 反雷对联(其一)}

\shanglian{Mono6}{投机取巧,误入八二六}
\xialian{Mono6}{兴妖作怪,枉活七十七}
\xialian{狗力大仙}{沽名钓誉,枉活七十七}

\section{145 反雷对联(其二)}

\shanglian{RobL61}{劝解不听,年老昏聩扮猴戏}
\xialian{雷绍夫}{真理未识,暮时残朽学犬鸣}

\section{146 反雷对联(其三)}

\shanglian{Mono6}{潜藏井底,武斗中拾得性命}
\xialian{雷绍夫}{暗出盒外,文争里丢掉脸皮}

\section{147 反雷对联(其四)}

\shanglian{RobL61}{藏字不识,大脑中全是歪理}
\xialian{Mono6}{咏雷尽录,思想里皆为邪说}

\section{148 反雷对联(其五)}

\shanglian{Mono6}{无知无德,运动力无人认可}
\xialian{雷绍夫}{有礼有节,反雷理有据依存}

\section{149 反雷对联(其六)}

\shanglian{乐山的雷公}{胸无点墨,泼猴竟敢研数理}
\xialian{雷绍夫}{腹缺诗书,老狗安能究诗文}

\section{150 反雷对联(其七)}

\shanglian{小野寺麗}{雷猿昏心聩智终成遗老}
\xialian{RobL61}{朽木败智无知后为疯猴}

\section{151 反雷对联(其八)}
\shanglian{雷绍夫}{峨眉山下,乐山水前,有猿猴通人言,\\ 泼妇骂街,狺狺狂吠,妄称其能究物理}
\xialian{乐山的雷公}{天府国中,大佛像边,有蜀犬能吠日,\\ 不学无术,日日胡言,竟云其将大道昭}

\section{152 反雷对联(其九)}
\shanglian{RobL61}{瞧瞧雷公:少智甚矣,科妄甚矣,狂犬吠日,可悲甚矣}
\xialian{Mono6,引绍武名句}{看看楼主:空虚之极,恐惧之极,泼妇骂街,无耻之极}

\section{153 反雷对联(其十)}
\shanglian{小野寺麗}{今朝猿陷埔墁招人摈}
\xialian{雷绍夫}{明日犬落井盖惹群嘲}

\section{154 反雷对联(其十一)}

\shanglian{小野寺麗}{今日犬彘匣中荡}
\xialian{Mono6}{明天猿猱线上行}

\section{155 反雷对联(其十二)}
\shanglian{小野寺麗}{十年沉浮,运力谬论吧间绱袖}
\xialian{Mono6}{一夜风波,南航笑谈群内流传}

\section{156 反雷对联(其十三)}
\shanglian{RobL61}{看无知猿猴摸线嚎啕啸}
\xialian{乐山的雷公}{观愚蠢老翁进盒呜咽啼}

\section{157 反雷对联(其十四)}
\shanglian{Уоgцгт}{猴起如厕厕动猴不动}
\xialian{小野寺麗}{犬伏进盒盒移犬不移}
~\\

\section{158 反雷对联(其十五)}
\shanglian{Mono6}{欲证加速四点九,老朽桥上扔石头, \\ 逐帧分析打脸,把视频一删,忙曰否}
\xialian{RobL61}{曾言公转五百万,泼猴群中扯慌言, \\ 几图佐证扇面\footnote{扇面:与“打脸”同义。},将傻论三申,自食言}

\newpage

\section{159 反雷对联(其十六)}
\shanglian{雷绍夫}{八二六老猴犯贱,民科谬论害人不浅,几时进盒摸零线}
\xialian{RobL61}{四点九傻论心寒,臆想邪说令吾感叹,今朝雷犬愈发憨}
\xialian{Mono6}{七十七愚翁发狂,歪理邪说出口无穷,明日入土败万年}
~\\

\section{160 反雷对联(其十七)}
\shanglian{小野寺麗}{雷于雨田进井盖}
\xialian{雷绍夫}{盒为合皿吓老猴}

\newpage

\section{161 反雷对联(其十八)}
\shanglian{RobL61}{零分之一,二货绍武,三年雷吧,加速四点玖; \\ 五(无)脑雷猴,六百藏头,七百暗讽,可悲八二六; \\ 自取九(咎)由,十年漫扮猴}

\xialian{Mono6}{十年也九(久),八方歪理,七旬老狗,辩论六神糊; \\ 五兆周期,四处否认,三番抵赖,无知二百五; \\ 不可一世,零线晚节输}
~\\

\section{162 反雷对联(其十九)}

\shanglian{Mono6}{赵公大德,真理将胜}
\xialian{RobL61,引经典藏字}{雷神弱智,民科必亡}

\newpage

\section{163 反雷对联(其二十)}

\shanglian{Mono6}{咏雷千余首,多少钓鱼辱骂}
\xialian{RobL61}{邪说逾百篇,大都颅臆谬谈}
\xialian{乐山的雷公}{藏字数百条,几分谩骂讥嘲}
~\\

\section{164 反雷对联(其二十一)}

\shanglian{RobL61}{不懂装懂大狂犬,笑孔明羽扇纶巾,真无可药救}
\xialian{乐山的雷公}{人云亦云老泼猴,嘲反雷不刊之论,绝贻笑大方}

\newpage

\section{165 反雷对联(其二十二)}

\shanglian{Mono6}{几只蠢猪,几条疯狗,不足为惧}
\xialian{雷公助我、Mono6}{一个老朽,一介愚夫,无脸而谈}
\xialian{街角,咏雷}{一碗开水,一个馒头,亦可成才}
~\\

\section{166 反雷对联(其二十三)}

\shanglian{Mono6}{动荡年间,唯有绍武猴入井}
\xialian{小野寺麗}{太平盛世,竟出为民牛弹琴}

\newpage

\section{167 反雷对联(其二十四)}

\shanglian{雷绍夫}{绍武妄言,恰似一条疯狗}
\xialian{雷绍夫}{反雷高论,宛如几朵奇葩}
\shanglian{雷绍夫、Mono6}{咏猴妄言,恰似两三疯狗}
\xialian{雷绍夫、Mono6}{反雷高论,宛如千万明灯}

\end{document}
    
    
